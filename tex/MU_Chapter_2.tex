\documentclass{article}
\usepackage{graphicx} % Required for inserting images
\usepackage{amsmath}
\usepackage{amssymb}

\title {Probability and Computing, 2nd Edition \\[2ex] \large Chapter 2 Solutions}
\author{Hahndeul Kim}
\date{June 2025}

\begin{document}

\maketitle
\newpage
\section*{2.1}
$\textbf{E}[X]=\left(\sum\limits_{i=1}^ki\right)/k=(k+1)/2$.
\section*{2.2}
The probability to type "proof" is $1/26^5$. As there are $1,000,000-5+1=999,996$ positions to start the word "proof", the desired probability would be $999996/26^5$ by the linearity of expectations.
\section*{2.3}
Take $f$ as $f(x)=-x^2$ and $X$ as a random variable with $\Pr(X=1)=\Pr(X=2)=1/2$. Then, $-5/2=\textbf{E}[f(X)]<f(\textbf{E}[X])=-9/4$.\\
Take $f$ as $f(x)=x$ and $X$ as above. Then, $\textbf{E}[f(X)]=f(\textbf{E}[X])=3/2$.\\
Take $f$ as $f(x)=x^2$ and $X$ as above. Then, $9/4=f(\textbf{E}[X])<\textbf{E}[f(X)]=5/2$.
\section*{2.4}
Take $f(x)=x^k$, which is convex when $k$ is an positive even integer.
Then by Jensen's inequality, $\textbf{E}[f(X)] \geq f(\textbf{E}[X])$ holds.
\section*{2.5}
Let the event that $X$ is even be $Y$. Then $\Pr(Y)=\sum\limits_{i=0,2,\cdots}\binom{n}{i}(\frac{1}{2})^n$ holds.\
As is known, $\sum\limits_{i=0,2,\cdots}\binom{n}{i}=2^{n-1}$, so $\Pr(Y)={\frac{1}{2}}$ is valid.
\section*{2.6}
(a) $X_1$ can be $2$, $4$ or $6$. Therefore $\textbf{E}[X|X_1$ is even$] = (3+4+\cdots+8)\times \frac{1}{18}+(5+6+\cdots+10)\times \frac{1}{18}+(7+8+\cdots+12)\times \frac{1}{18}=\frac{15}{2}$.\\
(b) $\textbf{E}[X|X_1 = X_2]=(2+4+6+8+10+12)\times\frac{1}{6}=7$.\\
(c) $\textbf{E}[X_1|X=9]=(3+4+5+6)\times\frac{1}{4}=\frac{9}{2}$.\\
(d) $\textbf{E}[X_1 - X_2|X=k]=0$, since $X_1$ and $X_2$ are independent dice rolls.
\section*{2.7}
(a) $\sum\limits_{k=1}^\infty p(1-p)^{k-1}q(1-q)^{k-1}=pq\cdot\frac{1}{1-(1-p)(1-q)}=\frac{pq}{p+q-pq}$.\\
(b) $\textbf{E}[\max (X,Y)]=\sum\limits_{k=1}^\infty \Pr(X\geq k$ or $Y \geq k)=\sum\limits_{k=1}^\infty(1-\Pr(X<k,Y<k))=\sum\limits_{k=1}^\infty(1-(1-(1-p)^{k-1})(1-(1-q)^{k-1}))$\\
$=\sum\limits_{k=1}^\infty((1-p)^{k-1}+(1-q)^{k-1}-(1-p)^{k-1}(1-q)^{k-1})=\frac{1}{p}+\frac{1}{q}-\frac{1}{p+q-pq}$.\\
(c) $\Pr(\min(X,Y)=k)=\Pr(X=k)\Pr(Y\geq k)+\Pr(Y=k)\Pr(X\geq k)-\Pr(X=Y=k)=(1-p)^{k-1}(1-q)^{k-1}(p+q-pq)=(1-(p+q-pq))^{k-1}(p+q-pq)$.\\
(d) $\textbf{E}[X|X\leq Y]=\textbf{E}[\min(X,Y)]=1/(p+q-pq)$, since $\min(X,Y)\sim Geom(p+q-pq)$ from the previous problem.
\section*{2.8}
(a) Expected number of girls: $\textbf{E}[G]=1\times\sum\limits_{i=1}^k(\frac{1}{2})^i=1-2^{-k}$.\\
Expected number of boys: $\textbf{E}[B]=(\frac{1}{2})^k\times k+\sum\limits_{i=1}^k(\frac{1}{2})^i\times(i-1)=\frac{2^k-1}{2^k}$.\\
(b) The number of total children now follows $Geom(1/2)$. Thus, $\textbf{E}[G+B]=2$ holds. Since $\textbf{E}[G]=\lim\limits_{k\rightarrow \infty}\frac{2^k-1}{2^k}=1$ holds using the result of the previous problem, $\textbf{E}[B]=1$.
\section*{2.9}
(a) $\textbf{E}[\max(X_1,X_2)]=\sum\limits_{i=1}^k\frac{i^2-(i-1)^2}{k^2}\times i=\frac{4k^2+3k-1}{6k}$.\\
$\textbf{E}[\min(X_1,X_2)]=\sum\limits_{i=1}^k\frac{(k+1-i)^2-(k-i)^2}{k^2}\times i=\frac{2k^2+3k+1}{6k}$.\\
(b) Since two dice are independent, $\textbf{E}[X_1]=\textbf{E}[X_2]=\frac{k+1}{2}$. Therefore, the claim holds.\\
(c) By the linearity of expectations, $\textbf{E}[\max(X_1,X_2)]+\textbf{E}[\min(X_1,X_2)]=\textbf{E}[\max(X_1,X_2)+\min(X_1,X_2)]$ holds.
Since $\{\max(X_1,X_2),\min(X_1,X_2)\}=\{X_1,X_2\}$, $\textbf{E}[\max(X_1,X_2)+\min(X_1,X_2)]=\textbf{E}[X_1+X_2]=\textbf{E}[X_1]+\textbf{E}[X_2]$ holds, again by the linearity of expectations.
Thus, the claim in the previous problem must be true.
\section*{2.10}
(a) Base case: when $n=1,2$, it is trivial from the definition of convexity.\\
Inductive step: Suppose that the claim holds for $n=k$.
Now, let $\sum\limits_{i=1}^{k+1}\lambda_i=1$ and $x_1,...,x_{k+1}\in \mathbb{R}$.
Then, by the definition of convexity,\\
$f(\sum\limits_{i=1}^{k+1}\lambda_ix_i)\leq(1-\lambda_{k+1})f(\frac{1}{1-\lambda_{k+1}}(\sum\limits_{i=1}^k\lambda_ix_i))+\lambda_{k+1}f(x_{k+1})$ holds.
Now, from the inductive hypothesis, $(1-\lambda_{k+1})f(\frac{1}{1-\lambda_{k+1}}(\sum\limits_{i=1}^k\lambda_ix_i))\leq(1-\lambda_{k+1})\sum\limits_{i=1}^k\frac{\lambda_i}{1-\lambda_{k+1}}f(x_i)=\sum\limits_{i=1}^{k}\lambda_if(x_i)$ holds.
Therefore, $f(\sum\limits_{i=1}^{k+1}\lambda_ix_i)\leq\sum\limits_{i=1}^{k+1}\lambda_if(x_i)$. $\blacksquare$\\
(b) If $X$ takes on only finitely many values, we can denote the set of possible values as $\{x_1,...,x_n\}$.
Then, since $\sum_i\Pr(X=x_i)=1$, $f(\sum\limits_{i=1}^n\Pr(X=x_i)x_i)\leq\sum\limits_{i=1}^n\Pr(X=x_i)f(x_i)$ holds from the previous problem.
This is equivalent to $f(\textbf{E}[X])\leq\textbf{E}[f(X)]$.
\section*{2.11}
Inductive proof.\\
Base case: It is trivial on $n=1$.\\
When $n=2$, $\textbf{E}[X_1+X_2|Y=y]=\sum\limits_i\sum\limits_j(i+j)\Pr(X_1=i,X_2=j|Y=y)$\\
$=\sum\limits_i\sum\limits_ji\Pr(X_1=i,X_2=j|Y=y)+\sum\limits_i\sum\limits_jj\Pr(X_1=i,X_2=j|Y=y)$.\\
Now, by the law of total probability, above equation is equivalent to\\
$\sum\limits_i i\Pr(X_1=i|Y=y)+\sum\limits_j j\Pr(X_2=j|Y=y)=\textbf{E}[X_1|Y=y]+\textbf{E}[X_2|Y=y]$.\\
Inductive step: Suppose that the claim holds for $n=k$. Then,\\
$\textbf{E}[\sum\limits_{i=1}^{k+1}X_i|Y=y]=\textbf{E}[X_{k+1}|Y=y]+\textbf{E}[\sum\limits_{i=1}^kX_i|Y=y]=\sum\limits_{i=1}^{k+1}\textbf{E}[X_i|Y=y]$. $\blacksquare$
\section*{2.12}
The expected number of cards to draw to see all $n$ cards is equivalent to the coupon collector's problem in the textbook.
Let $X_i$ be the number of draws to perform to observe the $i$th card.
Then $X_i\sim Geom(1-\frac{i-1}{n})$ holds, deriving $\textbf{E}[\sum\limits_{i=1}^nX_i]=\sum\limits_{i=1}^n\textbf{E}[X_i]=\sum\limits_{i=1}^n\frac{n}{n-i+1}=\sum\limits_{i=1}^n\frac{n}{i}$.\\
Let $Y_i$ be the indicator variable that is $1$ if $i$th card was not chosen within $2n$ draws.
Then the expected number of unchosen cards would be $\sum\limits_{i=1}^n\textbf{E}[Y_i]=n(\frac{n-1}{n})^{2n}$.\\
Using the same idea, the expected number of cards chosen only once would be $n\times\binom{2n}{1}\frac{1}{n}(\frac{n-1}{n})^{2n-1}$.
\section*{2.13}
(a) The problem is equivalent to the coupon collector's problem, since the probability of observing the $i$th coupon stays as $1-\frac{2i-2}{2n}=1-\frac{i-1}{n}$.\\
(b) For any positive integer $k$, the result is equivalent. The probability of observing the $i$th coupon is $1-\frac{ki-k}{kn}=1-\frac{i-1}{n}$.
\section*{2.14}
The $n$th flip must be head. Taking this into account, there would be $\binom{n-1}{k-1}$ ways to assign the ordering of $k-1$ heads and $n-k$ tails.\\
Therefore, $\Pr(X=n)=\binom{n-1}{k-1}p^k(1-p)^{n-k}$.
\section*{2.15}
Since it is inefficient to algebraically compute the expectation of a negative binomial distribution,
simply introduce $X_1,...,X_k$ where $X_i$ denotes the number of flips performed after $(i-1)$th head until $i$th head.
Then, $\textbf{E}[\sum\limits_{i=1}^kX_i]=\sum\limits_{i=1}^k\textbf{E}[X_i]=k/p$.
\section*{2.16}
(a) Take $n=2^k$, and let $X_i$ be an indicator variable that is $1$ if a streak of length $\log_2n+1=k+1$ occurred starting from the $i$th flip.\\
Then $\textbf{E}[\sum\limits_{i=1}^{n-k}X_i]=\sum\limits_{i=1}^{n-k}\textbf{E}[X_i]=(n-k)(\frac{1}{2})^k=1-\frac{\log_2n}{n}$ holds.\\
Now, $1-\frac{\log_2n}{n}$ is $1-o(1)$ since $\lim\limits_{n\rightarrow\infty}\frac{\log_2n}{n}=0$.\\
(b) Let $\lfloor \log_2 n - 2\log_2\log_2 n \rfloor=\delta$.
Note that the desired probability is upper-bounded by the probability that all disjoint $\delta$ blocks are not a streak, which is $(1-(\frac{1}{2})^{\delta-1})^{\lfloor n/\delta \rfloor}$.\\
$(1-(\frac{1}{2})^{\delta-1})^{\lfloor n/\delta \rfloor}\leq(1-(\frac{1}{2})^{\log_2n-2\log_2\log_2n})^{\lfloor n/\delta \rfloor}=(1-\frac{(\log_2n)^2}{n})^{\lfloor n/\delta \rfloor}$\\
$\leq(1-\frac{(\log_2n)^2}{n})^{n/\log_2n}\leq e^{-\log_2n}=n^{-\log_2e}\leq n^{-1}$ ($1-x\leq e^{-x}$). $\blacksquare$
\section*{2.17}
$\textbf{E}[Y_0]=1$, $\textbf{E}[Y_1]=2p$ obviously holds. Now, we have $\textbf{E}[Y_i|Y_{i-1}=j]=2pj$ for $i\geq 1$.
Then, by the definition of conditional expectation, $\textbf{E}[Y_i]=\textbf{E}[\textbf{E}[Y_i|Y_{i-1}]]=\sum_j2pj\Pr(Y_{i-1}=j)=2p\textbf{E}[Y_{i-1}]$.
Thus, $\textbf{E}[Y_i]=(2p)^i$, and the expected total number of copies $\textbf{E}[\sum\limits_{i=0}^\infty Y_i]$ is bounded if $p<1/2$.
\section*{2.18}
Inductive proof.\\
Base case: It is trivial on $n=1$.\\
Inductive step: Suppose that $\Pr(X_k=i)=1/k$ for all $i$ where $X_k$ is the item stored after the $k$th item appeared.\\
Then, $\Pr(X_{k+1}=i)=\Pr(X_k=i)\times(1-\frac{1}{k+1})=\frac{1}{k+1}$ for all $1\leq i \leq k$, and obviously $\Pr(X_{k+1}=k+1)=\frac{1}{k+1}$ which is the probability of replacement. $\blacksquare$
\section*{2.19}
Let $X_k$ be the item stored after the $k$th item appeared. Since $k=1$ is trivial, we will solve for $k\geq 2$.
Then $\Pr(X_k=i)=(\frac{1}{2})^{k+1-i}$ for all $2\leq i\leq k$ and $\Pr(X_k=1)=\Pr(X_k=2)$.
\section*{2.20}
Let $X_i$ be an indicator variable that is $1$ if $\pi(i)=i$.
Then the expected number of fixed points would be $\textbf{E}[\sum\limits_{i=1}^nX_i]=\sum\limits_{i=1}^n\textbf{E}[X_i]=n\times\frac{1}{n}$.
\section*{2.21}
$\textbf{E}[\sum\limits_{i=1}^{n} |a_i -i|] = \sum\limits_{i=1}^{n} \textbf{E}[|a_i - i|]=\sum\limits_{i=1}^n\sum\limits_{j=1}^n|j-i|$
$=\sum\limits_{i=1}^n\frac{1}{n}(\sum\limits_{j=1}^{i-1}j+\sum\limits_{j=1}^{n-i}j)$\\
$=\sum\limits_{i=1}^n\frac{1}{n}(i^2-i)=\frac{n^2-1}{3}$.
\section*{2.22}
In bubble sort, the number of all possible pairs $(i,j)$ that $a_i$ and $a_j$ are inverted is equivalent to the number of inversions that need to be corrected.\\
Let $X$ be the number of inversions.
Then $\textbf{E}[X]=\sum\limits_{i=1}^{n-1}\sum\limits_{j=i+1}^n\Pr(a_i>a_j)=\sum\limits_{i=1}^{n-1}\sum\limits_{j=i+1}^n\frac{1}{2}$,
since all numbers are distinct and the input is a random permutation.\\
Thus, $\textbf{E}[X]=\sum\limits_{i=1}^n\frac{1}{2}(n-i)=\frac{n(n-1)}{4}$.
\section*{2.23}
Let $X_i$ be the number of swaps needed for the $i$th element. Since the input is a random permutation, $\textbf{E}[X_i]=(i-1)/2$.\\
Thus, the expected number of swaps would be $\textbf{E}[\sum\limits_{i=1}^{n}X_i]=\sum\limits_{i=1}^n\textbf{E}[X_i]=\frac{n(n-1)}{4}$.
\section*{2.24}
Let $X$ be the number of dice rolls, and $X_1$ be the result of the first roll.\\
Then $\textbf{E}[X]=\textbf{E}[X|X_1=6]\Pr(X_1=6)+\textbf{E}[X+1]\Pr(X_1\neq 6)$ holds by the memoryless property.\\
Thus, $\textbf{E}[X]=\frac{1}{6}(\frac{1}{6}\times2+\frac{5}{6}\textbf{E}[X+2])+\frac{5}{6}\textbf{E}[X+1]=\frac{35}{36}\textbf{E}[X]+\frac{7}{6}$.
$\therefore \textbf{E}[X]=42$.
\section*{2.25}
(a) To make the test negative, all the people in the pool need to be negative, which happens with probability $(1-p)^k$. Thus, the desired probability is $1-(1-p)^k$.\\
(b) Since there are $n/k$ pools, the number of expected necessary tests would be $(n/k)\times((1-(1-p)^k)\times1+(1-p)^k\times(k+1))=n(1+\frac{1}{k}-(1-p)^k)$.\\
(c) Compute the derivative of the expectation derived in (b), and numerically solve the gradient being zero.\\
(d) $n(1+\frac{1}{k}-(1-p)^k)<n$ must hold for the pooling method to be better than naïve method. The inequality evaluates to $\frac{1}{k}<(1-p)^k$ for a fixed $k$.
\section*{2.26}
Let $X_i$ be the number of $i$-cycles in the graph. Then, the expected number of cycles would be $\textbf{E}[\sum\limits_{i=1}^nX_i]=\sum\limits_{i=1}^n\textbf{E}[X_i]$.\\
$\textbf{E}[X_i]=\binom{n}{i}\frac{(k-1)!}{n(n-1)\cdots(n-k+1)}=\frac{n!}{(n-i)!i!}\frac{(i-1)!(n-i)!}{n!}=\frac{1}{i}$ holds.
Thus, $\textbf{E}[\sum\limits_{i=1}^nX_i]=\sum\limits_{i=1}^n\frac{1}{i}=H(n)$ ($\textit{harnomic number}$).
\section*{2.27}
$\textbf{E}[X]=\sum\limits_{i=1}^\infty x\Pr(X=x)=\sum\limits_{i=1}^\infty(6/\pi^2)x^{-1}=\infty$,
which follows from the well-known divergence of harmonic series.
\section*{2.28}
If the player won at the $k$th spin for the first time, the total money lost is $(1+2+\cdots+2^{k-2})$, and earned money is $2^{k-1}$.
Since $(1+2+\cdots+2^{k-2})=2^{k-1}-1$, the player eventually wins a dollar.\\
$\textbf{E}[X]=\sum\limits_{i=1}^\infty(\frac{1}{2})^i(2^{i-1}-1)=\sum\limits_{i=1}^\infty(\frac{1}{2}-(\frac{1}{2})^i)=\infty$.
This implies that this strategy is impractical and would lead to bankruptcy, since the player has a finite amount of money.
\section*{2.29}

\end{document}
