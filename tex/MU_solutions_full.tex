\documentclass{article}
\usepackage{graphicx} % Required for inserting images
\usepackage{amsmath}
\usepackage{amssymb}
\usepackage{titling}
\renewcommand\maketitlehooka{\null\mbox{}\vfill}
\renewcommand\maketitlehookd{\vfill\null}


\font\myfont=cmr12 at 14.5pt
\title {\myfont Solutions to Probability and Computing (2nd Edition)}
\author{Hahndeul Kim}
\date{July 2025}

\begin{document}

\begin{titlingpage}
\maketitle
\end{titlingpage}
\newpage
\section{Events and Probability}
\subsection{}
(a) Choose five coins to be the heads. $\binom{10}{5} / {2^{10}} = 63/256$.\\
(b) Choose six or more coins to be the heads. $\sum\limits_{k=6}^{10}\binom{10}{k} / {2^{10}} = 193/512$.\\
(c) For each $\textit{i}$, the probability that the two flips are the same is $1/2$. Therefore, the desired probability is $1/32$.\\
(d) $\Pr(\text{4 consecutive heads})=139/2^{10}$, ruling out duplicate cases.\\
For more than 4 consecutive heads, the counting process is straightforward.\\
$\Pr(\text{5 consecutive heads})=64/2^{10}$,
$\Pr(\text{6 consecutive heads})=28/2^{10}$,\\
$\Pr(\text{7 consecutive heads})=12/2^{10}$,
$\Pr(\text{8 consecutive heads})=5/2^{10}$,\\
$\Pr(\text{9 consecutive heads})=2/2^{10}$,
$\Pr(\text{10 consecutive heads})=1/2^{10}$,\\
In summary, the desired probability is $251/1024$.
\subsection{}
(a) Choose a number to be same. $6/36 = 1/6$.\\
(b) By symmetry, $1/2 \times(1-1/6) = 5/12$.\\
(c) $(1+3+5+5+3+1)/36=1/2$.\\
(d) In addition to two identical rolls, $(1,4)$ and $(4,1)$ are the cases. $8/36 = 2/9$.
\subsection{}
(a) The probability that no ace is included is $\binom{48}{2}/\binom{52}{2}$. Thus, the desired probability is $1-\binom{48}{2}/\binom{52}{2}$.\\
(b) Similar to (a), the desired probability is $1-\binom{48}{5}/\binom{52}{5}$.\\
(c) (Deferred Decision) Match the rank of the first card drawn. $3/51$.\\
(d) $\binom{13}{5}/\binom{52}{5}$.\\
(e) Choose a rank to be triplets and choose another rank to be doublets, then build the triplets and doublets. The desired probability is $(13\times12\times\binom{4}{3}\binom{4}{2})/\binom{52}{5}$.
\subsection{}
If the loser has won $\textit{k}$ games, then the total number of games played would be $\textit{n}+\textit{k}$, and the winner must have won the last game.
Thus, there are $\binom{n+k-1}{k}$ ways to choose games that the loser won.\\
Let the random variable $X$ be the number of games won by the loser. Then $\Pr(X=k)=\binom{n+k-1}{k}/2^{n+k-1}$ holds.
\subsection{}
(a) (1,2), (1,4), (1,9), (6,2), (6,4), (6,9), (8,2), (8,4), (8,9)\\
Alice will win with probability $5/9$.\\
(b) (2,3), (2,5), (2,7), (4,3), (4,5), (4,7), (9,3), (9,5), (9,7)\\
Alice will win with probability $5/9$.\\
(c) (3,1), (3,6), (3,8), (5,1), (5,6), (5,8), (7,1), (7,6), (7,8)\\
Alice will win with probability $5/9$.
\subsection{}
Let the random variable $X_n$ be the number of white balls in the bin for a fixed $n$. The claim that $\Pr(X_n=k)=1/(n-1)$ for all $k$ will be inductively proved.\\
Base Case: If n=3, $\Pr(X_3 = 1)=\Pr(X_3 = 2)=1/2$.\\
Inductive Step: Suppose that $\Pr(X_n = k) = 1/(n-1)$. Then for $n+1$, $\Pr(X_{n+1}=1)=(1/2)\times(2/3)\times\cdots\times((n-1)/n)=1/n$, $\Pr(X_{n+1}=n)=(1/2)\times(2/3)\times\cdots\times((n-1)/n)=1/n$ holds.
For $2\leq k \leq n-1$, $\Pr(X_{n+1}=k)=\Pr(X_n=k)\times((n-k)/n)+\Pr(X_n=k-1)\times((k-1)/n)=1/n$. $\blacksquare$
\subsection{}
(a) Inductive proof.\\
Base case: $\Pr(E_1 \cup E_2) = \Pr(E_1) + \Pr(E_2) - \Pr(E_1 \cap E_2)$. (by the axioms of probability)\\
Inductive Step: Suppose that there are $n+1$ events $E_1$, $E_2$, $\cdots$, $E_{n+1}$, and $\Pr\left(\bigcup\limits_{i=1}^k E_i\right) = \sum\limits_{i=1}^k\Pr(E_i)-\sum\limits_{i<j}\Pr(E_i \cap E_j)+\cdots+(-1)^{k+1}\Pr\left(\bigcap\limits_{i=1}^k E_i \right)$ for all $k \leq n$.
Now, $\Pr\left(\bigcup\limits_{i=1}^{n+1} E_i\right) = \Pr\left(\left(\bigcup\limits_{i=1}^n E_i\right)\bigcup E_{n+1}\right)$$=\Pr\left(\bigcup\limits_{i=1}^n E_i \right) + \Pr(E_{n+1}) - \Pr\left(\bigcup\limits_{i=1}^n (E_i \cap E_{n+1})\right)$.\\
$\Pr\left(\bigcup\limits_{i=1}^{n+1} E_i\right)$$=\Pr\left(\bigcup\limits_{i=1}^n E_i \right) + \Pr(E_{n+1}) - \Pr\left(\bigcup\limits_{i=1}^n (E_i \cap E_{n+1})\right) = \sum\limits_{i=1}^{n+1}\Pr(E_i)-\sum\limits_{i<j \leq n}\Pr(E_i \cap E_j)+\cdots+(-1)^{n+1}\Pr\left(\bigcap\limits_{i=1}^n E_i \right)$\\$-\left(\sum\limits_{i=1}^n\Pr(E_i \cap E_{n+1})-\sum\limits_{i<j<n+1}\Pr(E_i \cap E_j \cap E_{n+1})+\cdots+(-1)^{n+1}\Pr\left(\bigcap\limits_{i=1}^{n+1} E_i \right)\right)$\\
= $\sum\limits_{i=1}^{n+1}\Pr(E_i)-\sum\limits_{i<j \leq n+1}\Pr(E_i \cap E_j)+\cdots+(-1)^{n+2}\Pr\left(\bigcap\limits_{i=1}^{n+1} E_i \right)$. $\blacksquare$\\
(b), (c) For all $\omega\in\Omega$, if $\omega$ is included in $k\geq 1$ of the $n$ given sets, then $\omega$ is counted once in LHS and $\sum\limits_{i=1}^\ell(-1)^{i+1}\binom{k}{i}$ in RHS.
Note that $\binom{k}{i}=0$ here if $k<i$.\\
Since $\sum\limits_{i=1}^\ell(-1)^{i+1}\binom{k}{i}-1=\sum\limits_{i=0}^\ell(-1)^{i+1}\binom{k}{i}=0$ if $k\leq \ell$,
we investigate the sign of $\sum\limits_{i=0}^\ell(-1)^{i+1}\binom{k}{i}$ in the case of $k>\ell$.
Then it can be easily shown that this is nonnegative when $\ell$ is odd and nonpositive when $\ell$ is even, using the properties of binomial coefficients.
\subsection{}
Let $D_n$ be the event that the chosen integer is divisible by $n$. Then the desired probability is $\Pr(D_4 \cup D_6 \cup D_9) = \Pr(D_4) + \Pr(D_6) + \Pr(D_9) - \Pr(D_4 \cap D_6) - \Pr(D_6 \cap D_9) - \Pr(D_4 \cap D_9) + \Pr(D_4 \cap D_6 \cap D_9) = \Pr(D_4) + \Pr(D_6) + \Pr(D_9) - \Pr(D_{12}) - \Pr(D_{18}) - \Pr(D_{36}) + \Pr(D_{36}) = 388889/1000000$.
\subsection{}
Let $S_i$ be the event that the $\log_2 n + k$ flips starting from the $i^{\text{th}}$ flip are consecutive heads. Then $\Pr(S_i) = (1/2)^{\log_2 n + k} = 1/2^kn$. By union bound, the desired probability $p$ is bounded as $p \leq (n-\log_2 n-k+1)/2^kn \leq 1/2^k$.
\subsection{}
Let A be the event that a fair coin is flipped, and B the event that a biased coin is flipped. Now, let X be the result of the flip. Then the desired probability is $\Pr(B|X=H)=\Pr(B\cap(X=H))/(\Pr(X=H|A)\Pr(A)+\Pr(X=H|B)\Pr(B))=0.5/(0.25+0.5)=2/3$.
\subsection{}
(a) The given probability indicates the cases where the bit is flipped even times. Flipping the bit even times is the necessary and sufficient condition to receive the correct bit.\\
(b) $\frac{1-q_1}{2} \times \frac{1+q_2}{2} + \frac{1+q_1}{2} \times \frac{1-q_2}{2} = \frac{1-q_1q_2}{2}$.\\
(c) Inductive proof.\\
Base case: when $n=1$, the probability of receiving the correct bit is $1-p$.\\
Inductive step: Suppose that the probability of receiving the correct bit after $k$ relays is $\frac{1+(1-2p)^k}{2}$. Then after $k+1$ relays, the probability is $\frac{1+(1-2p)^k}{2} \times (1-p) + \frac{1-(1-2p)^k}{2} \times p = \frac{1+(1-2p)^{k+1}}{2}$. $\blacksquare$
\subsection{}
Without loss of generality, assume that the contestant initially chooses the first door and Monty opens the second door. Then, $\Pr(C=1|O=2)$ and $\Pr(C=3|O=2)$ are to be compared.\\
$\Pr(C=1|O=2)=\frac{\Pr(O=2|C=1)\Pr(C=1)}{\Pr(O=2|C=1)\Pr(C=1)+\Pr(O=2|C=3)\Pr(C=3)}=\frac{1}{3}$\\
$\Pr(C=3|O=2)=\frac{\Pr(O=2|C=3)\Pr(C=3)}{\Pr(O=2|C=1)\Pr(C=1)+\Pr(O=2|C=3)\Pr(C=3)}=\frac{2}{3}$\\
Thus, the contestant should switch curtains.
\subsection{}
Let $D$ be the event that an individual has the disorder, and $R$ be the individual's test result.\\
$\Pr(D|R=P)=\frac{\Pr(R=P|D)\Pr(D)}{\Pr(R=P|D)\Pr(D)+\Pr(R=N|D)\Pr(D)} = \frac{0.999\times0.02}{0.999\times0.02+0.005\times0.98}\approx0.803$.
\subsection{}
Let $M$ be the result of the given match, $E_1$ the event that I am better, $E_2$ the event that both are equal and $E_3$ the event that the opponent is better.\\
The desired probability is $\Pr(E_3|M)=\frac{\Pr(M|E_3)\Pr(E_3)}{\Pr(M|E_1)\Pr(E_1)+\Pr(M|E_2)\Pr(E_2)+\Pr(M|E_3)\Pr(E_3)}=\frac{0.4\times0.6^3}{0.4^3\times0.6+0.5^4+0.4\times0.6^3}\approx0.461$.
\subsection{}
By the principle of deferred decisions, consider the situation where the last roll is left. Then, there is always a unique result of the last roll to make the final sum divisible by 6. Thus, the desired probability is $1/6$.
\subsection{}
(a) The desired probability is $6/6^3=1/36$.\\
(b) The desired probability is $(6\cdot5\cdot3)/6^3=5/12$.\\
(c) Under the given condition, the player will lose if he/she fails twice to roll the other die to match the other two dice. The desired probability is $1-(5/6)^2=11/36$.\\
(d) The desired probability is $\frac{1}{36}+\frac{5}{12}\times\frac{11}{36}+\frac{5}{9}\times(\frac{1}{36}+\frac{5}{12}\times\frac{1}{6}+\frac{5}{9}\times\frac{1}{36})=\frac{197}{972}$.
\subsection{}
If the vector $\overline{r}$ is chosen uniformly from $\{0,1,\cdots,k-1\}^n$, then\\
$\Pr(\textbf{AB}\overline{r}=\textbf{C}\overline{r})\leq 1/k$.\\
Thus, if the identity test is run p times, then the true positive probability is $1-\frac{1}{k^{p+1}+1}$.
\subsection{}
First, uniformly choose an integer $x$ from $\{0,\cdots,n-1\}$. Then evaluate $F(z)$ as $F(z)=F((z-x)+x)=(F(z-x)+F(x))\mod{m}$.
At the lookup table, each value $F(z-x)$ and $F(x)$ will be of the correct value with probability $4/5$.
Considering that we take modulo $m$, the output will be equal to $F(z)$ with probability at least $(4/5)^2=16/25$.\\
If the algorithm is repeated three times, then it is possible to take the majority if it exists, or otherwise simply take the first output.
For the algorithm to fail, at least two of the three outputs need to be wrong, and that occurs with probability less than $\binom{3}{2}\times(\frac{9}{25})^2\times\frac{16}{25}+\binom{3}{3}\times(\frac{9}{25})^3\approx0.295$.
Thus, the final output is correct with probability at least $0.705$.
\subsection{}
Let $A$ be the event that the sum of two dice rolls is $2$, and let $B$ be the event that the first roll is $2$. Then $0=\Pr(A|B)<\Pr(A)=1/36$.\\
Let $A$ be the event that a roll of a blue die is $1$, and let $B$ be the event that a roll of a red die is $1$. Then $\Pr(A|B)=\Pr(A)=1/36$.\\
Let $A$ be the event that the sum of two dice rolls is $2$, and let $B$ be the event that the first roll is $1$. Then $1/6=\Pr(A|B)>\Pr(A)=1/36$.
\subsection{}
The goal is to show that $\Pr\left(\bigcap\limits_{i\in I} \overline{E_i}\right)=\prod\limits_{i\in I}\Pr(\overline{E_i})$.\\
$\Pr\left(\bigcap\limits_{i\in I} \overline{E_i}\right)=\Pr\left(\overline{\bigcup\limits_{i\in I}E_i}\right) = 1-\Pr\left(\bigcup\limits_{i\in I}E_i\right)$\\
$=1-\left(\sum\limits_{i\in I}\Pr(E_i)-\sum\limits_{i<j\in I}\Pr(E_i\cap E_j)+\cdots+(-1)^{|I|+1}\Pr\left(\bigcap\limits_{i\in I}E_i\right)\right)$\\
$=1-\sum\limits_{i\in I}\Pr(E_i)+\sum\limits_{i<j\in I}\Pr(E_i)\Pr(E_j)+\cdots+(-1)^{|I|}\prod\limits_{i\in I}\Pr(E_i)$\\
$=\prod\limits_{i\in I}(1-\Pr(E_i))=\prod\limits_{i\in I}\Pr(\overline{E_i})$. $\blacksquare$
\subsection{}
Suppose that a fair four-sided die is rolled. Let $X=\{1,2\}$, $y=\{1,3\}$,$Z=\{1,4\}$. Then the events are pairwise independent but not mutually independent ($1/4=\Pr(X\cap Y\cap Z) \neq \Pr(X)\Pr(Y)\Pr(Z)=1/8$.
\subsection{}
(a) For any $X \subset \{1,\cdots,n\}$, each element in $X$ is chosen with probability $1/2$, and each element not in $X$ is dropped with probability $1/2$.
Thus, the probability of $X$ to be generated is $1/2^{|X|}\times1/2^{n-|X|}=1/2^n$.
As there are $2^n$ possible subsets, $X$ is equally likely to be any one of the possible subsets.\\
(b) For $X\subseteq Y$, each element in $X$ must be in $Y$, which happens with probability $3/4$. Thus, $\Pr(X\subseteq Y)=(3/4)^n$.\\
For $X\cup Y = \{1,\cdots,n\}$, each element in $\{1,\cdots,n\}$ must be in at least one of two sets $X$ and $Y$, which happens with probability $3/4$. Thus, $\Pr(X\cup Y = \{1,\cdots,n\})=(3/4)^n$.
\subsection{}
After executing the randomized min-cut algorithm, there can be at most $\binom{n}{2}$ edges between the two vertices in the reduced graph, as $n$ vertices are assumed.
As each edge is a candidate for min-cut sets, there can be at most $\binom{n}{2}=\frac{n(n-1)}{2}$ distinct min-cut sets.
\subsection{}
\subsection{}
\newpage
\section{Discrete Random Variables and Expectation}
\subsection{}
$\textbf{E}[X]=\left(\sum\limits_{i=1}^ki\right)/k=(k+1)/2$.
\subsection{}
The probability to type "proof" is $1/26^5$. As there are $1,000,000-5+1=999,996$ positions to start the word "proof", the desired probability would be $999996/26^5$ by the linearity of expectations.
\subsection{}
Take $f$ as $f(x)=-x^2$ and $X$ as a random variable with $\Pr(X=1)=\Pr(X=2)=1/2$. Then, $-5/2=\textbf{E}[f(X)]<f(\textbf{E}[X])=-9/4$.\\
Take $f$ as $f(x)=x$ and $X$ as above. Then, $\textbf{E}[f(X)]=f(\textbf{E}[X])=3/2$.\\
Take $f$ as $f(x)=x^2$ and $X$ as above. Then, $9/4=f(\textbf{E}[X])<\textbf{E}[f(X)]=5/2$.
\subsection{}
Take $f(x)=x^k$, which is convex when $k$ is an positive even integer.
Then by Jensen's inequality, $\textbf{E}[f(X)] \geq f(\textbf{E}[X])$ holds.
\subsection{}
Let the event that $X$ is even be $Y$. Then $\Pr(Y)=\sum\limits_{i=0,2,\cdots}\binom{n}{i}(\frac{1}{2})^n$ holds.\
As is known, $\sum\limits_{i=0,2,\cdots}\binom{n}{i}=2^{n-1}$, so $\Pr(Y)={\frac{1}{2}}$ is valid.
\subsection{}
(a) $X_1$ can be $2$, $4$ or $6$. Therefore $\textbf{E}[X|X_1$ is even$] = (3+4+\cdots+8)\times \frac{1}{18}+(5+6+\cdots+10)\times \frac{1}{18}+(7+8+\cdots+12)\times \frac{1}{18}=\frac{15}{2}$.\\
(b) $\textbf{E}[X|X_1 = X_2]=(2+4+6+8+10+12)\times\frac{1}{6}=7$.\\
(c) $\textbf{E}[X_1|X=9]=(3+4+5+6)\times\frac{1}{4}=\frac{9}{2}$.\\
(d) $\textbf{E}[X_1 - X_2|X=k]=0$, since $X_1$ and $X_2$ are independent dice rolls.
\subsection{}
(a) $\sum\limits_{k=1}^\infty p(1-p)^{k-1}q(1-q)^{k-1}=pq\cdot\frac{1}{1-(1-p)(1-q)}=\frac{pq}{p+q-pq}$.\\
(b) $\textbf{E}[\max (X,Y)]=\sum\limits_{k=1}^\infty \Pr(X\geq k$ or $Y \geq k)=\sum\limits_{k=1}^\infty(1-\Pr(X<k,Y<k))=\sum\limits_{k=1}^\infty(1-(1-(1-p)^{k-1})(1-(1-q)^{k-1}))$\\
$=\sum\limits_{k=1}^\infty((1-p)^{k-1}+(1-q)^{k-1}-(1-p)^{k-1}(1-q)^{k-1})=\frac{1}{p}+\frac{1}{q}-\frac{1}{p+q-pq}$.\\
(c) $\Pr(\min(X,Y)=k)=\Pr(X=k)\Pr(Y\geq k)+\Pr(Y=k)\Pr(X\geq k)-\Pr(X=Y=k)=(1-p)^{k-1}(1-q)^{k-1}(p+q-pq)=(1-(p+q-pq))^{k-1}(p+q-pq)$.\\
(d) $\textbf{E}[X|X\leq Y]=\textbf{E}[\min(X,Y)]=1/(p+q-pq)$, since $\min(X,Y)\sim Geom(p+q-pq)$ from the previous exercise.
\subsection{}
(a) Expected number of girls: $\textbf{E}[G]=1\times\sum\limits_{i=1}^k(\frac{1}{2})^i=1-2^{-k}$.\\
Expected number of boys: $\textbf{E}[B]=(\frac{1}{2})^k\times k+\sum\limits_{i=1}^k(\frac{1}{2})^i\times(i-1)=\frac{2^k-1}{2^k}$.\\
(b) The number of total children now follows $Geom(1/2)$. Thus, $\textbf{E}[G+B]=2$ holds. Since $\textbf{E}[G]=\lim\limits_{k\rightarrow \infty}\frac{2^k-1}{2^k}=1$ holds using the result of the previous exercise, $\textbf{E}[B]=1$.
\subsection{}
(a) $\textbf{E}[\max(X_1,X_2)]=\sum\limits_{i=1}^k\frac{i^2-(i-1)^2}{k^2}\times i=\frac{4k^2+3k-1}{6k}$.\\
$\textbf{E}[\min(X_1,X_2)]=\sum\limits_{i=1}^k\frac{(k+1-i)^2-(k-i)^2}{k^2}\times i=\frac{2k^2+3k+1}{6k}$.\\
(b) Since two dice are independent, $\textbf{E}[X_1]=\textbf{E}[X_2]=\frac{k+1}{2}$. Therefore, the claim holds.\\
(c) By the linearity of expectations, $\textbf{E}[\max(X_1,X_2)]+\textbf{E}[\min(X_1,X_2)]=\textbf{E}[\max(X_1,X_2)+\min(X_1,X_2)]$ holds.
Since $\{\max(X_1,X_2),\min(X_1,X_2)\}=\{X_1,X_2\}$, $\textbf{E}[\max(X_1,X_2)+\min(X_1,X_2)]=\textbf{E}[X_1+X_2]=\textbf{E}[X_1]+\textbf{E}[X_2]$ holds, again by the linearity of expectations.
Thus, the claim in the previous exercise must be true.
\subsection{}
(a) Base case: when $n=1,2$, it is trivial from the definition of convexity.\\
Inductive step: Suppose that the claim holds for $n=k$.
Now, let $\sum\limits_{i=1}^{k+1}\lambda_i=1$ and $x_1,...,x_{k+1}\in \mathbb{R}$.
Then, by the definition of convexity,\\
$f(\sum\limits_{i=1}^{k+1}\lambda_ix_i)\leq(1-\lambda_{k+1})f(\frac{1}{1-\lambda_{k+1}}(\sum\limits_{i=1}^k\lambda_ix_i))+\lambda_{k+1}f(x_{k+1})$ holds.
Now, from the inductive hypothesis, $(1-\lambda_{k+1})f(\frac{1}{1-\lambda_{k+1}}(\sum\limits_{i=1}^k\lambda_ix_i))\leq(1-\lambda_{k+1})\sum\limits_{i=1}^k\frac{\lambda_i}{1-\lambda_{k+1}}f(x_i)=\sum\limits_{i=1}^{k}\lambda_if(x_i)$ holds.
Therefore, $f(\sum\limits_{i=1}^{k+1}\lambda_ix_i)\leq\sum\limits_{i=1}^{k+1}\lambda_if(x_i)$. $\blacksquare$\\
(b) If $X$ takes on only finitely many values, we can denote the set of possible values as $\{x_1,...,x_n\}$.
Then, since $\sum_i\Pr(X=x_i)=1$, $f(\sum\limits_{i=1}^n\Pr(X=x_i)x_i)\leq\sum\limits_{i=1}^n\Pr(X=x_i)f(x_i)$ holds from the previous exercise.
This is equivalent to $f(\textbf{E}[X])\leq\textbf{E}[f(X)]$.
\subsection{}
Inductive proof.\\
Base case: It is trivial on $n=1$.\\
When $n=2$, $\textbf{E}[X_1+X_2|Y=y]=\sum\limits_i\sum\limits_j(i+j)\Pr(X_1=i,X_2=j|Y=y)$\\
$=\sum\limits_i\sum\limits_ji\Pr(X_1=i,X_2=j|Y=y)+\sum\limits_i\sum\limits_jj\Pr(X_1=i,X_2=j|Y=y)$.\\
Now, by the law of total probability, above equation is equivalent to\\
$\sum\limits_i i\Pr(X_1=i|Y=y)+\sum\limits_j j\Pr(X_2=j|Y=y)=\textbf{E}[X_1|Y=y]+\textbf{E}[X_2|Y=y]$.\\
Inductive step: Suppose that the claim holds for $n=k$. Then,\\
$\textbf{E}[\sum\limits_{i=1}^{k+1}X_i|Y=y]=\textbf{E}[X_{k+1}|Y=y]+\textbf{E}[\sum\limits_{i=1}^kX_i|Y=y]=\sum\limits_{i=1}^{k+1}\textbf{E}[X_i|Y=y]$. $\blacksquare$
\subsection{}
The expected number of cards to draw to see all $n$ cards is equivalent to the coupon collector's problem in the textbook.
Let $X_i$ be the number of draws to perform to observe the $i$th card.
Then $X_i\sim Geom(1-\frac{i-1}{n})$ holds, deriving $\textbf{E}[\sum\limits_{i=1}^nX_i]=\sum\limits_{i=1}^n\textbf{E}[X_i]=\sum\limits_{i=1}^n\frac{n}{n-i+1}=\sum\limits_{i=1}^n\frac{n}{i}$.\\
Let $Y_i$ be the indicator variable that is $1$ if $i$th card was not chosen within $2n$ draws.
Then the expected number of unchosen cards would be $\sum\limits_{i=1}^n\textbf{E}[Y_i]=n(\frac{n-1}{n})^{2n}$.\\
Using the same idea, the expected number of cards chosen only once would be $n\times\binom{2n}{1}\frac{1}{n}(\frac{n-1}{n})^{2n-1}$.
\subsection{}
(a) The exercise is equivalent to the coupon collector's problem, since the probability of observing the $i$th coupon stays as $1-\frac{2i-2}{2n}=1-\frac{i-1}{n}$.\\
(b) For any positive integer $k$, the result is equivalent. The probability of observing the $i$th coupon is $1-\frac{ki-k}{kn}=1-\frac{i-1}{n}$.
\subsection{}
The $n$th flip must be head. Taking this into account, there would be $\binom{n-1}{k-1}$ ways to assign the ordering of $k-1$ heads and $n-k$ tails.\\
Therefore, $\Pr(X=n)=\binom{n-1}{k-1}p^k(1-p)^{n-k}$.
\subsection{}
Since it is inefficient to algebraically compute the expectation of a negative binomial distribution,
simply introduce $X_1,...,X_k$ where $X_i$ denotes the number of flips performed after $(i-1)$th head until $i$th head.
Then, $\textbf{E}[\sum\limits_{i=1}^kX_i]=\sum\limits_{i=1}^k\textbf{E}[X_i]=k/p$.
\subsection{}
(a) Take $n=2^k$, and let $X_i$ be an indicator variable that is $1$ if a streak of length $\log_2n+1=k+1$ occurred starting from the $i$th flip.\\
Then $\textbf{E}[\sum\limits_{i=1}^{n-k}X_i]=\sum\limits_{i=1}^{n-k}\textbf{E}[X_i]=(n-k)(\frac{1}{2})^k=1-\frac{\log_2n}{n}$ holds.\\
Now, $1-\frac{\log_2n}{n}$ is $1-o(1)$ since $\lim\limits_{n\rightarrow\infty}\frac{\log_2n}{n}=0$.\\
(b) Let $\lfloor \log_2 n - 2\log_2\log_2 n \rfloor=\delta$.
Note that the desired probability is upper-bounded by the probability that all disjoint $\delta$ blocks are not a streak, which is $(1-(\frac{1}{2})^{\delta-1})^{\lfloor n/\delta \rfloor}$.\\
$(1-(\frac{1}{2})^{\delta-1})^{\lfloor n/\delta \rfloor}\leq(1-(\frac{1}{2})^{\log_2n-2\log_2\log_2n})^{\lfloor n/\delta \rfloor}=(1-\frac{(\log_2n)^2}{n})^{\lfloor n/\delta \rfloor}$\\
$\leq(1-\frac{(\log_2n)^2}{n})^{n/\log_2n}\leq e^{-\log_2n}=n^{-\log_2e}\leq n^{-1}$ ($1-x\leq e^{-x}$). $\blacksquare$
\subsection{}
$\textbf{E}[Y_0]=1$, $\textbf{E}[Y_1]=2p$ obviously holds. Now, we have $\textbf{E}[Y_i|Y_{i-1}=j]=2pj$ for $i\geq 1$.
Then, by the definition of conditional expectation, $\textbf{E}[Y_i]=\textbf{E}[\textbf{E}[Y_i|Y_{i-1}]]=\sum_j2pj\Pr(Y_{i-1}=j)=2p\textbf{E}[Y_{i-1}]$.
Thus, $\textbf{E}[Y_i]=(2p)^i$, and the expected total number of copies $\textbf{E}[\sum\limits_{i=0}^\infty Y_i]$ is bounded if $p<1/2$.
\subsection{}
Inductive proof.\\
Base case: It is trivial on $n=1$.\\
Inductive step: Suppose that $\Pr(X_k=I_i)=1/k$ for all $i$ where $X_k$ is the item stored after the $k$th item ($I_k$) appeared.\\
Then, $\Pr(X_{k+1}=I_i)=\Pr(X_k=I_i)\times(1-\frac{1}{k+1})=\frac{1}{k+1}$ for all $1\leq i \leq k$, and obviously $\Pr(X_{k+1}=I_{k+1})=\frac{1}{k+1}$ which is the probability of replacement. $\blacksquare$
\subsection{}
Let $X_k$ be the item stored after the $k$th item appeared. Since $k=1$ is trivial, we will solve for $k\geq 2$.
Then $\Pr(X_k=i)=(\frac{1}{2})^{k+1-i}$ for all $2\leq i\leq k$ and $\Pr(X_k=1)=\Pr(X_k=2)$.
\subsection{}
Let $X_i$ be an indicator variable that is $1$ if $\pi(i)=i$.
Then the expected number of fixed points would be $\textbf{E}[\sum\limits_{i=1}^nX_i]=\sum\limits_{i=1}^n\textbf{E}[X_i]=n\times\frac{1}{n}$.
\subsection{}
$\textbf{E}[\sum\limits_{i=1}^{n} |a_i -i|] = \sum\limits_{i=1}^{n} \textbf{E}[|a_i - i|]=\sum\limits_{i=1}^n\sum\limits_{j=1}^n|j-i|$
$=\sum\limits_{i=1}^n\frac{1}{n}(\sum\limits_{j=1}^{i-1}j+\sum\limits_{j=1}^{n-i}j)$\\
$=\sum\limits_{i=1}^n\frac{1}{n}(i^2-i)=\frac{n^2-1}{3}$.
\subsection{}
In bubble sort, the number of all possible pairs $(i,j)$ that $a_i$ and $a_j$ are inverted is equivalent to the number of inversions that need to be corrected.\\
Let $X$ be the number of inversions.
Then $\textbf{E}[X]=\sum\limits_{i=1}^{n-1}\sum\limits_{j=i+1}^n\Pr(a_i>a_j)=\sum\limits_{i=1}^{n-1}\sum\limits_{j=i+1}^n\frac{1}{2}$,
since all numbers are distinct and the input is a random permutation.\\
Thus, $\textbf{E}[X]=\sum\limits_{i=1}^n\frac{1}{2}(n-i)=\frac{n(n-1)}{4}$.
\subsection{}
Let $X_i$ be the number of swaps needed for the $i$th element. Since the input is a random permutation, $\textbf{E}[X_i]=(i-1)/2$.\\
Thus, the expected number of swaps would be $\textbf{E}[\sum\limits_{i=1}^{n}X_i]=\sum\limits_{i=1}^n\textbf{E}[X_i]=\frac{n(n-1)}{4}$.
\subsection{}
Let $X$ be the number of dice rolls, and $X_1$ be the result of the first roll.\\
Then $\textbf{E}[X]=\textbf{E}[X|X_1=6]\Pr(X_1=6)+\textbf{E}[X+1]\Pr(X_1\neq 6)$ holds by the memoryless property.\\
Thus, $\textbf{E}[X]=\frac{1}{6}(\frac{1}{6}\times2+\frac{5}{6}\textbf{E}[X+2])+\frac{5}{6}\textbf{E}[X+1]=\frac{35}{36}\textbf{E}[X]+\frac{7}{6}$.
$\therefore \textbf{E}[X]=42$.
\subsection{}
(a) To make the test negative, all the people in the pool need to be negative, which happens with probability $(1-p)^k$. Thus, the desired probability is $1-(1-p)^k$.\\
(b) Since there are $n/k$ pools, the number of expected necessary tests would be $(n/k)\times((1-(1-p)^k)\times1+(1-p)^k\times(k+1))=n(1+\frac{1}{k}-(1-p)^k)$.\\
(c) Compute the derivative of the expectation derived in (b), and numerically solve the gradient being zero.\\
(d) $n(1+\frac{1}{k}-(1-p)^k)<n$ must hold for the pooling method to be better than naïve method. The inequality evaluates to $\frac{1}{k}<(1-p)^k$ for a fixed $k$.
\subsection{}
Let $X_i$ be the number of $i$-cycles in the graph. Then, the expected number of cycles would be $\textbf{E}[\sum\limits_{i=1}^nX_i]=\sum\limits_{i=1}^n\textbf{E}[X_i]$.\\
$\textbf{E}[X_i]=\binom{n}{i}\frac{(k-1)!}{n(n-1)\cdots(n-k+1)}=\frac{n!}{(n-i)!i!}\frac{(i-1)!(n-i)!}{n!}=\frac{1}{i}$ holds.
Thus, $\textbf{E}[\sum\limits_{i=1}^nX_i]=\sum\limits_{i=1}^n\frac{1}{i}=H(n)$ ($\textit{harnomic number}$).
\subsection{}
$\textbf{E}[X]=\sum\limits_{i=1}^\infty x\Pr(X=x)=\sum\limits_{i=1}^\infty(6/\pi^2)x^{-1}=\infty$,
which follows from the well-known divergence of harmonic series.
\subsection{}
If the player won at the $k$th spin for the first time, the total money lost is $(1+2+\cdots+2^{k-2})$, and earned money is $2^{k-1}$.
Since $(1+2+\cdots+2^{k-2})=2^{k-1}-1$, the player eventually wins a dollar.\\
$\textbf{E}[X]=\sum\limits_{i=1}^\infty(\frac{1}{2})^i(2^{i-1}-1)=\sum\limits_{i=1}^\infty(\frac{1}{2}-(\frac{1}{2})^i)=\infty$.
This implies that this strategy is impractical and would lead to bankruptcy, since the player has a finite amount of money.
\subsection{}
Let $S_n=\sum\limits_{j=0}^nX_j$.
Then from the linearity of expectations for a finite number of random variables, $\textbf{E}[S_n]=\sum\limits_{j=0}^n\textbf{E}[X_j]$ holds.
Here, RHS converges from the given absolute convergence, and thus LHS should also converge.
Thus, applying $\lim\limits_{n\rightarrow\infty}$ on each side, we get $\textbf{E}[\sum\limits_{j=0}^\infty X_j]=\sum\limits_{j=0}^\infty\textbf{E}[X_j]$.
\subsection{}
Since a player needs to lose all previous $j-1$ bets in order to participate in the $j$th bet,
$\textbf{E}[X_j]=(1-(\frac{1}{2})^{j-1}) \times 0 + (\frac{1}{2})^j\times2^{j-1} + (\frac{1}{2})^j\times(-2^{j-1})=0$ holds.\\
$\sum\limits_{j=0}^\infty \textbf{E}[X_j]=0$ holds, thus the linearity of expectations does not hold here.\\
This exercise does not fall under the circumstances of exercise 2.29, since $\sum\limits_{j=0}^\infty\textbf{E}[|X_j|]=\infty$ holds.
\subsection{}
The expected winnings would be $\sum\limits_{k=1}^\infty(\frac{1}{2})^k \times \frac{2^k}{k} = \sum\limits_{k=1}^\infty \frac{1}{k} = \infty$. Thus, the player should be willing to pay any amount of money to play the game.
\subsection{}
(a) By definition, $\Pr(E_i)=0$ for $i \leq m$, and $\Pr(E)=\sum\limits_{i=1}^n\Pr(E_i)$ is true.\\
If $i>m$, then the $i$th candidate must be the best among all $n$ candidates, and the second-best candidate must be one of the first $m$ candidates.
Thus, $\Pr(E_i)=\frac{1}{n}\times\frac{m}{i-1}$.\\
Therefore, $\Pr(E)=\sum\limits_{i=m+1}^n\frac{1}{n}\times\frac{m}{i-1}=\frac{m}{n}\sum\limits_{j=m+1}^n\frac{1}{j-1}$.\\
(b) Since $\sum\limits_{j=m+1}^n\frac{1}{j-1} \geq \int_{m+1}^{n+1}\frac{1}{x-1}dx=\ln n-\ln m$, $\Pr(E) \geq \frac{m}{n}(\ln n-\ln m)$ holds.\\
Also, since $\sum\limits_{j=m+1}^n\frac{1}{j-1} \leq \int_{m}^{n}\frac{1}{x-1}dx=\ln (n-1)-\ln (m-1)$, $\Pr(E) \geq \frac{m}{n}(\ln (n-1)-\ln (m-1))$ holds.\\
(c) For a fixed $n$, $\frac{\partial}{\partial m} \frac{m(\ln n - \ln m)}{n} = \frac{\ln n - \ln m - 1}{n} = 0$ when $m=n/e$.
This choice of $m$ is the maximizer, since the given formula has only one local maximum w.r.t. $m$.\\
Since $\frac{m(\ln n - \ln m)}{n}=1/e$ when $m=n/e$, $\Pr(E) \geq 1/e$ holds by (b).
\newpage
\section{Moments and Deviations}
\subsection{}
$\textbf{E}[X^2]=\sum\limits_{i=1}^n\frac{1}{n}\times i^2 = \frac{(n+1)(2n+1)}{6}$, and $\textbf{E}[X]=\frac{n+1}{2}$.\\
Thus, $\textbf{Var}[X]=\textbf{E}[X^2]-(\textbf{E}[X])^2=\frac{n^2-1}{12}$.
\subsection{}
$\textbf{E}[X]=0$, and $\textbf{E}[X^2]=\sum\limits_{i=1}^k \frac{2}{2k+1} \times i^2 = \frac{k(k+1)}{3}$.\\
Thus, $\textbf{Var}[X]=\textbf{E}[X^2]-(\textbf{E}[X])^2=\frac{k(k+1)}{3}$.
\subsection{}
The variance of a single die roll is $\frac{35}{12}$ from problem 3.1.
Since all rolls are independent, $\Pr(|X-350| \geq 50) \leq \frac{1}{50^2}\times \frac{35}{12} \times 100 = \frac{7}{60}$.
\subsection{}
$\textbf{Var}[cX]=\textbf{E}[(cX-\textbf{E}[cX])^2]=\textbf{E}[c^2X^2 - 2cX\textbf{E}[cX]+(\textbf{E}[cX])^2]$\\
$=c^2(\textbf{E}[X^2] - (\textbf{E}[X])^2)=c^2\textbf{Var}[X]$. $\blacksquare$
\subsection{}
$\textbf{Var}[X-Y]=\textbf{E}[((X-Y)-\textbf{E}[X-Y])^2]=\textbf{E}[((X-\textbf{E}[X])-(Y-\textbf{E}[Y]))^2]$\\
$=\textbf{E}[(X-\textbf{E}[X])^2] - 2\textbf{E}[(X-\textbf{E}[X])(Y-\textbf{E}[Y])]+\textbf{E}[(Y-\textbf{E}[Y])^2]$\\
$=\textbf{Var}[X]-\textbf{Cov}[X,Y]+\textbf{Var}[Y]=\textbf{Var}[X]+\textbf{Var}[Y]$ ($X \perp\!\!\!\perp Y$). $\blacksquare$
\subsection{}
Let $X_i$ ($1 \leq i \leq k$) be the number of flips after $(i-1)$th head until $i$th head.
Since all flips are independent, the desired variance could be computed as $\sum\limits_{i=1}^k \textbf{Var}[X_i]$.\\
As $X_i\sim Geom(p)$, $\textbf{Var}[X_i]=(1-p)/p^2$ for all $i$. Thus, the desired variance is $k(1-p)/p^2$.
\subsection{}
Let $X$ be the number of increases. Then $\Pr(X=k)=\binom{d}{k}p^k(1-p)^{d-k}$.\\
Let the price of the stock after $d$ days be $V$.\\
Then $\textbf{E}[V]=\sum\limits_{k=0}^d qr^k(\frac{1}{r})^{d-k}\binom{d}{k}p^k(1-p)^{d-k}=\sum\limits_{k=0}^dq\binom{d}{k}(pr)^k(\frac{1-p}{r})^{n-k}$.\\
Let $M=pr + (1-p)/r = (1-p+pr^2)/r$. Then\\
$\textbf{E}[V]=M^d\sum\limits_{k=0}^dq\binom{d}{k}(\frac{pr}{M})^k(\frac{1-p}{rM})^{d-k}=M^d\sum\limits_{k=0}^dq\binom{d}{k}(\frac{pr^2}{rM})^k(\frac{1-p}{rM})^{d-k}=M^dq$.\\
Now we compute $\textbf{E}[V^2]=\sum\limits_{k=0}^dq^2 r^{2k}(\frac{1}{r})^{2d-2k}\binom{d}{k}p^k(1-p)^{d-k}$.\\
$\textbf{E}[V^2]=q^2\sum\limits_{k=0}^d\binom{d}{k}(pr^2)^k(\frac{1-p}{r^2})^{d-k}=q^2\left(pr^2+\frac{1-p}{r^2} \right)^d$ (similar to $\textbf{E}[V]$).\\
Thus, $\textbf{Var}[V]=q^2 \left((pr^2 + \frac{1-p}{r^2})^d - (pr + \frac{1-p}{r})^{2d}\right)$.\\
By plugging $q=1$ in, we get the desired result.
\subsection{}
Let $X$ be the running time of the given algorithm on input strings of size $n$.
Now, let $M$ be the longest running time of the algorithm among the input strings of size $n$. Then $\Pr(X\geq M) \geq 1/2^n$ by definition.\\
By Markov's inequality, $1/2^n \leq \Pr(X\geq M) \leq \frac{\textbf{E}[X]}{M}$, which leads to $M \leq 2^n \textbf{E}[X]$.
Since $\textbf{E}[X]=O(n^2)$, we get $M=O(n^2 2^n)$.
\subsection{}
(a) By linearity of expectations, $\textbf{E}[X^2]=\textbf{E}[\sum\limits_{i=1}^nX_i X]=\sum\limits_{i=1}^n\textbf{E}[X_iX]$.\\
Since $X_i$ are Bernoulli random variables, $\textbf{E}[X_iX]=\Pr(X_i=0)\times 0+\Pr(X_i=1)\times \textbf{E}[X|X_i=1]$. $\blacksquare$\\
(b) Using the equation proven in (a), $\textbf{E}[X^2]=\sum\limits_{i=1}^np\times(1+(n-1)p)=np+n(n-1)p^2$.
Thus, $\textbf{Var}[X]=\textbf{E}[X^2]-(\textbf{E}[X])^2=np+n(n-1)p^2-n^2p^2=np(1-p)$.
\subsection{}
Let $X\sim Geom(p)$, and let $Y=1$ if and only if $X=1$ and $Y=0$ otherwise. Then, by Lemma 2.5,
$\textbf{E}[X^3]=\Pr(Y=1)\textbf{E}[X^3|Y=1]+\Pr(Y=0)\textbf{E}[X^3|Y=0]$\\
$=p\times 1 + (1-p)\times \textbf{E}[X^3|Y=0]=p+(1-p)\times\textbf{E}[X^3|X>1]$.\\
Now, by the memoryless property of geometric distributions, $\textbf{E}[X^3|X>1]=\textbf{E}[(X+1)^3]$.
Thus, $\textbf{E}[X^3]=p+(1-p)(\textbf{E}[X^3]+3\textbf{E}[X^2]+3\textbf{E}[X]+1)$.\\
This leads to $\textbf{E}[X^3]=(p^2-6p+6)/p^3$.\\
Similarly, we can find $\textbf{E}[X^4]=(-p^3+14p^2-36p+24)/p^4$.
\subsection{}
Let $X=\sum\limits_{i<j} X_{i,j}$, where $X_{i,j}$ is an indicator variable that is $1$ if $a_i$ and $a_j$ are inverted. Then, we compute $\textbf{E}[X^2]$ as:\\
$\textbf{E}[X^2]=\textbf{E}[\sum\limits_{i<j}X_{i,j}^2 +\sum\limits_{|\{i,j,k,l\}|=4}X_{i,j}X_{k,l} + \sum\limits_{i<j<k}2(X_{i,j}X_{j,k}+X_{i,j}X_{i,k}+X_{i,k}X_{j,k})]$
$=\binom{n}{2}\cdot\frac{1}{2}+\binom{n}{4}\cdot\binom{4}{2}\cdot\frac{1}{4}+2\cdot\binom{n}{3}\cdot\left( \frac{1}{6}+\frac{1}{3}+\frac{1}{3}\right)=\frac{n(n-1)(9n^2-5n+10)}{144}$.\\
Since $\textbf{E}[X]=\frac{n(n-1)}{4}$, $\textbf{Var}[X]=\textbf{E}[X^2]-(\textbf{E}[X])^2=\frac{n(n-1)(2n+5)}{72}$.
\subsection{}
Let $X$ be a random variable with probability distribution like $\Pr(X=n)=\frac{1}{\zeta (3)n^3}$.\\
Then $\textbf{E}[X]=\sum\limits_{n=1}^\infty n\times\frac{1}{\zeta (3)n^3}=\frac{1}{\zeta(3)}\times\frac{\pi^2}{6}<\infty$.\\
However, since $\textbf{E}[X^2]=\sum\limits_{n=1}^\infty n^2\times\frac{1}{\zeta (3)n^3}=\sum\limits_{n=1}^\infty \frac{1}{n}\times\frac{1}{\zeta (3)}=\infty$ (harmonic series), the variance of $X$ is unbounded.
\subsection{}
Let $X$ be a random variable such that $\Pr(X=n)=\frac{1}{\zeta (k+2)n^{k+2}}$.\\
Then, similar to problem 3.12, $\textbf{E}[X^k]$ converges and $\textbf{E}[X^{k+1}]$ diverges.
\subsection{}
$\textbf{Var}[\sum\limits_{i=1}^nX_i]=\textbf{E}\left[\left(\sum\limits_{i=1}^n(X_i-\textbf{E}[X_i])\right)^2\right]$
$=\sum\limits_{i=1}^n\textbf{Var}[X_i]+\sum\limits_{i\neq j}\textbf{Cov}[X_i,X_j]=\sum\limits_{i=1}^n\textbf{Var}[X_i]+2\sum\limits_{i=1}^n\sum\limits_{i<j}\textbf{Cov}[X_i,X_j]$. $\blacksquare$
\subsection{}
$\textbf{E}[X_iX_j]=\textbf{E}[X_i]\textbf{E}[X_j]$ indicates that $\textbf{Cov}[X_i,X_j]=\textbf{E}[X_iX_j]-\textbf{E}[X_i]\textbf{E}[X_j]=0$.\\
Thus, $\textbf{Var}[X]=\sum\limits_{i=1}^n\textbf{Var}[X_i]$ holds.
\subsection{}
Suppose that we want the expectation to be $\mu$.
Then the desired $X$ should satisfy $\Pr(X=k\mu)=1/k$ and $\Pr(X=0)=1-1/k$.
\subsection{}
Suppose that we want the expectation to be $\mu$.
Then in order to satisfy $\Pr(|X-\textbf{E}[X]|\geq a)=\frac{\textbf{Var}[X]}{a^2}$, $X$ should satisfy:\\
$\Pr(X=\mu)=p$, $\Pr(X=\mu+a)=(1-p)/2$ and $\Pr(X=\mu-a)=(1-p)/2$.
\subsection{}
(a) $\Pr(X-\textbf{E}[X]\geq t\sigma[X])=\Pr[t(X-\textbf{E}[X])+\sigma[X]\geq(t^2+1)\sigma[X]]$\\
$\leq\Pr[(t(X-\textbf{E}[X])+\sigma[X])^2\geq(t^2+1)^2\textbf{Var}[X]]$\\
$\leq \textbf{E}[(t(X-\textbf{E}[X])+\sigma[X])^2]/(t^2+1)^2\textbf{Var}[X]$ (Markov's inequality)\\
=$(t^2\textbf{Var}[X]+\textbf{Var}[X])/(t^2+1)^2\textbf{Var}[X]=1/(t^2+1)$. $\blacksquare$\\
(b) Since probabilities cannot be greater than $1$, we only consider $t \geq 1$.\\
By Chebyshev's inequality, $\Pr(|X-\textbf{E}[X]| \geq t\sigma[X]) \leq 1/t^2 \leq 2/(1+t^2)$. $\blacksquare$
\subsection{}
(i) If $\mu=m$, then the claim is trivially valid.\\
(ii) If $\mu < m$, then let $t=|\mu-m|/\sigma$. Then using the result of exercise 3.18(a), $\Pr(X-\mu \geq |\mu-m|) \leq \frac{\sigma^2}{(\mu-m)^2+\sigma^2}$.\\
Now, from $1/2 \leq \Pr(X\geq m) \leq \Pr(X-\mu\geq|\mu-m|) \leq \frac{\sigma^2}{(\mu-m)^2+\sigma^2}$,
we conclude $(\mu-m)^2 \leq \sigma^2$, thus $|\mu-m| \leq \sigma$.\\
(iii) If $\mu > m$, then let $t=|\mu-m|/\sigma$.
Now, substituting $X$ into $-X$ in the result of exercise 3.18(a), we get $\Pr(-X + \textbf{E}[X] \geq t\sigma[X]) \leq 1/(t^2+1)$.\\
This leads to $\Pr(-X+\mu \geq |\mu-m|) \leq \frac{\sigma^2}{(\mu-m)^2+\sigma^2}$.\\
Similarly, from $1/2 \leq \Pr(X\leq m) \leq \Pr(-X+\mu\geq|\mu-m|) \leq \frac{\sigma^2}{(\mu-m)^2+\sigma^2}$,
we conclude $(\mu-m)^2 \leq \sigma^2$, thus $|\mu-m| \leq \sigma$. $\blacksquare$
\subsection{}
We first prove $\Pr(Y\neq 0) \leq \textbf{E}[Y]$. This is trivial since\\
$\textbf{E}[Y]-\Pr(Y\neq 0)=\sum\limits_{i=1}^\infty i\Pr(Y=i)-\sum\limits_{i=1}^\infty \Pr(Y=i)=\sum\limits_{i=1}^\infty(i-1)\Pr(Y=i)\geq 0$.\\
Now we prove $\frac{\textbf{E}[Y]^2}{\textbf{E}[Y^2]} \leq \Pr(Y\neq 0)$.
Let $X=Y|Y\neq 0$ such that $\Pr(X=x)=\Pr(Y=x|Y\neq 0)$. Since $(\textbf{E}[X])^2 \leq \textbf{E}[X^2]$, $\textbf{E}[Y|Y\neq 0]^2 \leq \textbf{E}[Y^2|Y\neq 0]$ holds.\\
Now, consider that $\textbf{E}[X]=\textbf{E}[0]\Pr(X=0)+\textbf{E}[X|X\neq0]\Pr(X\neq 0)=\textbf{E}[X|X\neq0]\Pr(X\neq 0)$ is valid for any random variable $X$.\\
Combined with $\textbf{E}[Y|Y\neq 0]^2 \leq \textbf{E}[Y^2|Y\neq 0]$,
$\left(\frac{\textbf{E}[Y]}{\Pr(Y\neq 0)}\right)^2 \leq \frac{\textbf{E}[Y^2]}{\Pr(Y\neq 0)}$ holds. $\blacksquare$
\subsection{}
(a) Let $Y=|X-\textbf{E}[X]|$. Then by Markov's inequality,\\
$\Pr(Y>t\sqrt[\leftroot{-2}\uproot{2}k]{\textbf{E}[Y^k]}) =\Pr(Y^k > t^k\textbf{E}[Y^k]) \leq \Pr(Y^k \geq t^k\textbf{E}[Y^k])\leq1/t^k$.\\
(b) If $k$ is odd, then $Y^k \neq (X-\textbf{E}[X])^k$ and $(X-\textbf{E}[X])^k$ may not always be positive.
Therefore, we cannot apply Markov's inequality in this case.
\subsection{}
Let $X_i$ be indicator variables that are $1$ if $\pi(i)=i$. Then $X_i\sim Bernoulli(1/n)$.\\
Then $\textbf{Var}[\sum\limits_{i=1}^nX_i]=\sum\limits_{i=1}^n\textbf{Var}[X_i]+2\sum\limits_{i=1}^n\sum\limits_{i<j}\textbf{Cov}[X_i,X_j]$.\\
Now, since $X_i\sim Bernoulli(1/n)$, $\textbf{Var}[X_i]=\frac{1}{n}\left(1-\frac{1}{n}\right)$.\\
$\textbf{Cov}[X_i,X_j]=\textbf{E}[X_iX_j]-\textbf{E}[X_i]\textbf{E}[X_j]$. Since $\textbf{E}[X_iX_j]=\frac{1}{n(n-1)}$, $\textbf{Cov}[X_i,X_j]=\frac{1}{n(n-1)}-\frac{1}{n^2}$.\\
Thus, $\textbf{Var}[\sum\limits_{i=1}^nX_i]=1-\frac{1}{n}+n(n-1)\left(\frac{1}{n(n-1)}-\frac{1}{n^2}\right)=1$.
\subsection{}
(a) Since each coin is fair, the pair of coins to decide the value of $Y_i$ can be one of $(H,T),(T,H),(H,H),(T,T)$ with equal probability. Thus, $\Pr(Y_i=0)=\Pr(Y_i=1)=1/2$.\\
(b) Let the $i$th coin be denoted as $C_i$.
Then if the first pair is $C_1,C_2$, the second pair is $C_1,C_3$ and the third pair is $C_2,C_3$,\\
$\Pr(Y_1=Y_2=Y_3=0\neq1/8=\Pr(Y_1=1)\Pr(Y_2=1)\Pr(Y_3=1)$.\\
(c) Let $i$th pair be $C_a,C_b$ and $j$th pair be $C_c,C_d$.\\
If $|\{a,b,c,d\}|=4$, then $Y_i$ and $Y_j$ are independent. The claim trivially holds.\\
If $|\{a,b,c,d\}|=3$, then $\textbf{E}[Y_iY_j]=\Pr(Y_i=Y_j=1)=1/4=\textbf{E}[X_i]\textbf{E}[X_j]$.\\
(d) $\textbf{Var}[Y]=\textbf{Var}[\sum\limits_{i=1}^mY_i]=\sum\limits_{i=1}^m\textbf{Var}[Y_i]$ holds from the result of exercise 3.15.\\
Since $\textbf{Var}[Y_i]=1/4$, $\textbf{Var}[Y]=\frac{n(n-1)}{8}$.\\
(e) By Chebyshev's inequality, $\Pr(|Y-\textbf{E}[Y]| \geq n) \leq \frac{\textbf{Var}[Y]}{n^2}=\frac{n-1}{8n}\leq\frac{1}{8}$.
\subsection{}
\subsection{}
\subsection{}
By Chebyshev's inequality, $\Pr\left(\left|\frac{X_1+X_2+\cdots+X_n}{n}-\mu\right|>\epsilon\right) \leq \frac{\textbf{Var}[\sum\limits_{i=1}^nX_i/n]}{\epsilon^2}=\frac{\sigma^2}{n\epsilon^2}$.\\
Since $0\leq\lim\limits_{n\rightarrow\infty}\Pr\left(\left|\frac{X_1+X_2+\cdots+X_n}{n}-\mu\right|>\epsilon\right)\leq\lim\limits_{n\rightarrow\infty}\frac{\sigma^2}{n\epsilon^2}=0$, the desired result is obtained by the squeeze theorem.
\newpage
\section{Chernoff and Hoeffding bounds}
\subsection{}
Let the number of games that Alice wins be $X$, where $X\sim B(n,0.6)$. Alice will lose the tournament with probability $\Pr(X\leq \frac{n-1}{2})$.
Now, let $\delta$ s. t. $(1-\delta) \times \frac{3n}{5} = \frac{n-1}{2}$ to obtain the tightest bound.\\
$\Pr(X\leq \frac{n-1}{2})=\Pr(X\leq(1-\delta)\textbf{E}[X])\leq \exp(-\frac{3n}{5}\cdot \delta^2 \cdot \frac{1}{2})$\\
$=\exp(-\frac{1}{10}(\frac{1}{12}n+\frac{5}{6}+\frac{25}{12n}))\leq \exp(-\frac{1}{8})$ (AM-GM inequality).
\subsection{}
With Markov's inequality, $\Pr(X\geq n/4) \leq (n/6)/(n/4)=2/3$.\\
With Chebyshev's inequality, $\Pr(X\geq n/4) \leq \Pr(|X-n/6|\geq n/12)\leq \frac{\textbf{Var}[X]}{(n/12)^2}$\\
$=\frac{144}{n^2}\times(n\cdot \frac{1}{6}\cdot\frac{5}{6})=20/n$.\\
To use Chernoff bounds, let $\delta=1/2$. Then $\Pr(X\geq n/4)=\Pr(X\geq(1+\delta)\textbf{E}[X])$\\
$\leq\left(\frac{e^{0.5}}{1.5^{1.5}}\right)^{n/6}=\left(\frac{e}{1.5^3}\right)^{n/12}$.
\subsection{}
(a) Let $X\sim B(n,p)$. Then $M_X(t)=\textbf{E}[e^{tX}]=\sum\limits_{i=0}^ne^{it}\Pr(X=i)$\\
$=\sum\limits_{i=0}^ne^{it}\binom{n}{i}p^i(1-p)^{n-i}=\sum\limits_{i=0}^n\binom{n}{i}(pe^t)^i(1-p)^{n-i}=(pe^t+1-p)^n$.\\
(b) $M_{X+Y}(t)=\textbf{E}[e^{t(X+Y)}]=\textbf{E}[e^{tX}e^{tY}]=\textbf{E}[e^{tX}]\textbf{E}[e^{tY}]=(pe^t+1-p)^{m+n}$.\\
(c) Since moment generating function uniquely determines the distribution, $X+Y\sim B(m+n,p)$.
\subsection{}
Let the total number of heads be $X$, where $X\sim B(100,\frac{1}{2})$. Then we find $\Pr(X\geq55)\approx0.1841$.\\
From Chernoff bound, we find that $\Pr(X\geq (1+\frac{1}{10})50)\leq \exp(-\frac{50}{3}\cdot\frac{1}{10^2})=\exp(-\frac{1}{6})\approx0.8465$.\\
For $Y\sim B(1000,\frac{1}{2})$, $\Pr(Y\geq 550)\approx0.0009$.\\
From Chernoff bound, we find that $\Pr(Y\geq (1+\frac{1}{10})500)\leq \exp(-\frac{500}{3}\cdot\frac{1}{10^2})=\exp(-\frac{5}{3})\approx0.1889$.\\
\subsection{}
Let $Y=NX$, so that we aim to satisfy $\Pr(|Y-Np|>N\epsilon p) \leq \delta$. Consider that\\
$\Pr(Y>Np(1+\epsilon)) < \exp(-Np\cdot \frac{\epsilon^2}{3})$, and $\Pr(Y<Np(1-\epsilon)) < \exp(-Np\cdot \frac{\epsilon^2}{2})$.\\
Thus, we aim to satisfy $\exp(-Np\cdot \frac{\epsilon^2}{3})+\exp(-Np\cdot \frac{\epsilon^2}{2}) \leq 2\exp(-Np\cdot \frac{\epsilon^2}{3}) \leq \delta$.\\
$\therefore N \geq \frac{3}{p\epsilon^2} \ln \frac{2}{\delta}$. With $\epsilon=0.1$, $\delta=0.05$ and $0.2 \leq p \leq 0.8$, $N\geq1500\ln40\approx5533$.
\subsection{}
(a) Let $X\sim B(1000000, 0.02)$. Then $\Pr(X\geq 40000) \leq e^{-20000/3}$.\\
(b) Set $X$ and $Y$ as given and choose $k$, $l$ such that $l \leq k - 10000$ so that bounding $\Pr((X>k)\cap(Y<l))$ suffices.
As examples, we choose $k=15300$ and $l=4900$ here.
Since $X\sim B(510000, 0.02)$, $Y\sim B(490000,0.02)$ and $X \perp\!\!\!\perp Y$, $\Pr((X>k)\cap(Y<l)) = \Pr(X>k)\Pr(Y<l) \leq e^{-10200/12}\times e^{-9800/8} = e^{-2025}$.
\subsection{}
Recall that $M_X(t)=\prod\limits_{i=1}^n (p_ie^t + (1-p_i)) = \prod\limits_{i=1}^n (1+p_i(e^t-1))\leq \prod\limits_{i=1}^ne^{p_i(e^t-1)}$\\
$=e^{\mu(e^t-1)}$ holds when $X$ is the sum of Poisson trials ($\Pr(X_i=1)=p_i$).\\
Let $t=\ln (1+\delta)$ and follow the derivation of Chernoff bounds.\\
$\Pr(X\geq (1+\delta)\mu_H) \leq \frac{\textbf{E}[e^{tX}]}{e^{t(1+\delta)\mu_H}}\leq\frac{e^{\mu(e^t-1)}}{e^{t(1+\delta)\mu_H}}$
$\leq \left( \frac{e^{e^t-1}}{e^{t(1+\delta)}}\right)^{\mu_H}=\left( \frac{e^\delta}{(1+\delta)^{(1+\delta)}}\right)^{\mu_H}$.\\
Similarly, let $t=\ln (1-\delta)$ and prove the latter inequality.\\
$\Pr(X\leq (1-\delta)\mu_L) \leq \frac{\textbf{E}[e^{tX}]}{e^{t(1-\delta)\mu_L}}\leq\frac{e^{\mu(e^t-1)}}{e^{t(1-\delta)\mu_L}}$
$\leq \left( \frac{e^{e^t-1}}{e^{t(1-\delta)}}\right)^{\mu_L}=\left( \frac{e^{-\delta}}{(1-\delta)^{(1-\delta)}}\right)^{\mu_L}$. $\blacksquare$
\subsection{}
For any permutation $\pi$ produced with the given approach, $\Pr(f=\pi)=\prod\limits_{i=1}^n\frac{1}{k+1-i}$ holds.
Since the number of possible permutations is $\frac{k!}{(k-n)!}=\frac{1}{\Pr(f=\pi)}$,
the given approach produces a permutation chosen uniformly at random from all permutations.\\
Now, let $X_j$ be the number of black box calls to determine $f(j)$. Then $X_j\sim Geom(\frac{k+1-j}{k})$ holds.
Thus, $\textbf{E}[\sum\limits_{i=1}^nX_i]=\sum\limits_{i=1}^n\textbf{E}[X_i]=\sum\limits_{i=1}^n\frac{k}{k+1-i}$.\\
When $k=n$, $\textbf{E}[\sum\limits_{i=1}^nX_i]=\sum\limits_{i=1}^n\frac{n}{i}=nH(n)\approx n \ln n$.\\
Similarly, when $k=2n$, $\textbf{E}[\sum\limits_{i=1}^nX_i]=\sum\limits_{i=1}^n\frac{2n}{n+i}=2n(H(2n)-H(n))\approx2n\ln2$. In this case, $\frac{2n+1-j}{2n}\geq\frac{2n+1-n}{2n}\geq\frac{1}{2}$.\\
Now, to derive the desired Chernoff bound, we first compute the moment generating function of $X=\sum\limits_{i=1}^nX_j$.
Let $p_i=\frac{2n+1-i}{2n}$. Since $X_i$ are independent, $\textbf{E}[e^{tX}]=\prod\limits_{i=1}^n\textbf{E}[e^{tX_i}]$
$=\prod\limits_{i=1}^n\left(\prod\limits_{j=1}^\infty(e^{tj}p_i(1-p_i)^{j-1}\right)=\prod\limits_{i=1}^n\left(\frac{p_i}{1-p_i}\prod\limits_{j=1}^\infty(e^t(1-p_i))^j\right)$.\\
Suppose that we choose $t$ s. t. $0<t<\ln2$ when deriving the Chernoff bound.\\ Then $\textbf{E}[e^{tX}]=\prod\limits_{i=1}^n\frac{p_ie^t}{1-e^t(1-p_i)}$.
Since $t>0$, $\frac{\partial}{\partial p_i}\left(\frac{p_ie^t}{1-e^t(1-p_i)}\right)=\frac{1-e^t}{(1-e^t(1-p_i))^2}<0$.
This leads to $\textbf{E}[e^{tX}]=\prod\limits_{i=1}^n\frac{p_ie^t}{1-e^t(1-p_i)}\leq \left(\frac{\frac{1}{2}e^t}{1-\frac{1}{2}e^t}\right)^n$.\\
Now derive the desired Chernoff bound with $\Pr(X\geq 4n) \leq \frac{\textbf{E}[e^{tX}]}{e^{4nt}}\leq\left(\frac{1}{(2-e^t)e^{3t}}\right)^n$.\\
Since the function $(2-e^t)e^{3t}$ has its maximum at $t=\ln\frac{3}{2}$ and $0<\ln\frac{3}{2}<\ln2$, we choose $t=\ln\frac{3}{2}$ for the tightest possible bound.\\
The desired bound would be $\Pr(X\geq 4n) \leq \left(\frac{1}{(2-e^t)e^{3t}}\right)^n \Big|_{t=\ln\frac{3}{2}}=(\frac{16}{27})^n$.
\subsection{}
(a) By Chebyshev's inequality, $\Pr[|\sum\limits_{i=1}^tX_i-\textbf{E}[X]|\geq\epsilon\textbf{E}[X]]\leq\frac{\textbf{Var}[X]}{t(\epsilon\textbf{E}[X])^2}$
$=\frac{r^2}{t\epsilon^2}$. Thus, setting $t$ to satisfy $\frac{r^2}{t\epsilon^2}\leq \delta$ suffices.
This leads to $t \geq \frac{r^2}{\epsilon^2\delta}$, which proves the claim.\\
(b) Set $\delta=1-3/4=1/4$. Then we get $t \geq \frac{4r^2}{\epsilon^2}$, which proves the claim.\\
(c) Let $Y_i$ be indicator variables that are $1$ if $|X_i-\textbf{E}[X]|\geq\epsilon\textbf{E}[X]$. Then let the median of $Y_i$s be $m$, and bound the probability $\Pr(|m-\textbf{E}[X]|\geq\epsilon\textbf{E}[X])$.\\
Note that $\textbf{E}[\sum\limits_{i=1}^tY_i]\leq t/4$ by definition, and $|m-\textbf{E}[X]|\geq\epsilon\textbf{E}[X]$ holds only if $\sum\limits_{i=1}^tY_i \geq t/2$.
Then, $\Pr(|m-\textbf{E}[X]|\geq\epsilon\textbf{E}[X]) \leq \Pr\left(\sum\limits_{i=1}^tY_i \geq t/2\right)$. Let $Y=\sum\limits_{i=1}^tY_i$.
Then $\Pr(Y\geq t/2)=\Pr\left(Y\geq(1+(\frac{t}{2\textbf{E}[Y]}-1))\textbf{E}[Y]\right)$\\
$\leq \left(\frac{e^\delta}{(1+\delta)^{1+\delta}}\right)^{\textbf{E}[Y]}=(\frac{2e}{t})^{t/2}\times e^{-\textbf{E}[Y]}\textbf{E}[Y]^{t/2}$.\\
Since $\frac{\partial}{\partial \textbf{E}[Y]}\left( (\frac{2e}{t})^{t/2}e^{-\textbf{E}[Y]}\textbf{E}[Y]^{t/2} \right)=(\frac{2e}{t})^{t/2} e^{-\textbf{E}[Y]}\textbf{E}[Y]^{t/2-1}(t/2-\textbf{E}[Y])> 0$,\\
substitute $t/4$ for $\textbf{E}[Y]$ to derive our bound. Thus, $\Pr(Y\geq t/2) \leq (\frac{e}{4})^{t/4}$.\\
Here we need $t$ that satisfies $(\frac{e}{4})^{t/4} \leq \delta$, which leads to $t \geq \frac{4}{\ln \frac{4}{e}}\ln \frac{1}{\delta}$.
Therefore, together with 4.9.(b), we only need $O(\log(1/\delta))$ estimates constructed from $O(r^2\log(1/\delta)/\epsilon^2)$ samples.
\subsection{}
Let $X=\sum\limits_{i=1}^{1000000}X_i$ where $X_i$ denotes the winnings of the $i$th game.\\
Then by the Chernoff bound, $\Pr(X\geq10000) \leq \frac{\textbf{E}[e^{tX}]}{e^{10000t}}=\left(\frac{\textbf{E}[e^{tX_i}]^{100}}{e^{t}}\right)^{10000}$\\
$=\left(\frac{(167/200)e^{-t}+(4/25)e^{2t}+(1/200)e^{99t}}{e^{0.01t}}\right)^{1000000}$.
Using graph software, you can choose $t=0.0006$ and derive $\Pr(X\geq10000) \leq 0.0001606$.
\subsection{}
Since $\textbf{E}[X_i]=1$, $\textbf{E}[X]=n$. Thus, we bound $\Pr(X\geq (1+\delta)n)$ as\\
$\Pr(X\geq (1+\delta)n)\leq\frac{\textbf{E}[e^{tX}]}{e^{t(1+\delta)n}}$ with $t>0$. $\textbf{E}[e^{tX}]=\prod\limits_{i=1}^n\textbf{E}[e^{tX_i}]=\left(\frac{1}{3}(1+e^t+e^{2t})\right)^n$\\
leads to $\Pr(X\geq (1+\delta)n)\leq\left(\frac{1+e^t+e^{2t}}{3e^{t(1+\delta)}}\right)^n$.
Although $t=\frac{\delta+\sqrt{4-3\delta^2}}{1-\delta}$ minimizes $\frac{1+e^t+e^{2t}}{3e^{t(1+\delta)}}$, it is too complex to be used as a generalized bound.
Thus, we put $t=\ln(1+\delta)$ for simplicity and derive $\Pr(X\geq (1+\delta)n)\leq\left(\frac{3+3\delta+\delta^2}{3(1+\delta)^{(1+\delta)}}\right)^n$.\\
The Chernoff bound for $\Pr(X\leq(1-\delta)n)$ can also be derived in a similar way.
\subsection{}
(a) We can think of $X_i$ as the number of tails between $i-1$th head and $i$th head.
Now, let $Y_i$ be indicator variables that are $1$ if $i$th flip is head.
Then let $Y=\sum\limits_{i=1}^{(1+\delta)2n}Y_i$, and derive the Chernoff bound as $\Pr(X\geq(1+\delta)2n)=\Pr(Y\leq n)$.
Since $\textbf{E}[Y]=(1+\delta)n$, $\Pr(Y\leq n)=\Pr(Y\leq (1-\frac{\delta}{1+\delta})\textbf{E}[Y])\leq e^{-\frac{1}{2}\textbf{E}[Y](\frac{\delta}{1+\delta})^2}=e^{-\frac{n\delta^2}{2(1+\delta)}}$.\\
(b) Here, the moment generating function for $X$ can be derived as $\textbf{E}[e^{tX}]=\left(\frac{e^t}{2-e^t}\right)^n$ for $0 < t < \ln 2$ (refer to the solution for exercise 4.8).\\
Thus, $\Pr(X\geq(1+\delta)2n)\leq\frac{(\frac{e^t}{2-e^t})^n}{e^{t(1+\delta)2n}}=\left(\frac{1}{e^{t(1+2\delta)}(2-e^t)}\right)^n$.
Since $e^{t(1+2\delta)}(2-e^t)$ is maximized at $t=\ln(\frac{1+2\delta}{1+\delta}) < \ln 2$, we choose it to derive the tightest bound.
Therefore, $\Pr(X\geq(1+\delta)2n)\leq\left((\frac{1+\delta}{1+2\delta})^{1+2\delta}(1+\delta)\right)^n$.\\
(c) To compare two bounds, we inspect the sign of $e^{-\frac{\delta^2}{2(1+\delta)}}-(\frac{1+\delta}{1+2\delta})^{1+2\delta}(1+\delta)$.
The simpler equivalent would be $(1+2\delta)\ln(1+2\delta)-(2+2\delta)\ln(1+\delta)-\frac{\delta^2}{2(1+\delta)}$.\\
The computation can be performed numerically using $\ln(x+1)=\sum\limits_{n=1}^\infty\frac{(-1)^n}{n}x^n$, or with the help of graph software.
In either way, it can be shown that the bound derived in (b) is better.
\subsection{}
(a) From the Chernoff bound, $\Pr(X\geq xn) \leq \textbf{E}[e^{tX}]/e^{txn}$.\\
Since $\textbf{E}[e^{tX}]=\prod\limits_{i=1}^n\textbf{E}[e^{tX_i}]=(1-p+pe^t)^n$, $\Pr(X\geq xn) \leq \left( \frac{1-p+pe^t}{e^{xt}}\right)^n=((1-p)e^{-xt}+pe^{(1-x)t})^n$.
To derive the tightest bound, we solve for $\frac{\partial}{\partial t} ((1-p)e^{-xt}+pe^{(1-x)t})=0$, which gives $t=\ln(x(1-p))-\ln((1-x)p)$.
Since $(1-p)e^{-xt}+pe^{(1-x)t}$ is convex w. r. t. $t$ with given conditions, this gives the minimum.
By plugging this in, we can show that $\Pr(X\geq xn) \leq e^{-nF(x,p)}$.\\
(b) Since $\frac{\partial^2}{\partial x^2} (F(x,p)-2(x-p)^2)=\frac{1}{x}+\frac{1}{1-x}-4=\frac{(2x-1)^2}{x(1-x)}\geq 0$, $F(x,p)-2(x-p)^2$ is convex w. r. t. $x$ when $0<x,p<1$.
Considering that $\frac{\partial}{\partial x}(F(x,p)-2(x-p)^2)=0$ yields $x=p$ and $(F(x,p)-2(x-p)^2) \Big|_{x=p}=0$, we get $F(x,p)-2(x-p)^2 \geq 0$.\\
(c) $\Pr(X\geq(p+\epsilon)n)\leq e^{-nF(p+\epsilon,p)}$ holds by (a), and $e^{-nF(p+\epsilon,p)} \leq e^{-n\times 2(p+\epsilon-p)^2}=e^{-2n\epsilon^2}$ holds by (b).\\
(d) Take $Y_i=1-X_i$, and let $Y=n-X$. Then, $\Pr(X\leq (p-\epsilon)n)=\Pr(Y\geq((1-p)+\epsilon)n)\leq e^{-2n\epsilon^2}$ holds by (c).
Combined with (c), we get $\Pr(|X-pn|\geq \epsilon n)=\Pr(X\leq(p-\epsilon)n)+\Pr(X\geq(p+\epsilon)n)\leq 2e^{-2n\epsilon^2}$.
\subsection{}
We first bound the moment generating function of $X$ to $\textbf{E}[e^{tX}]=\prod\limits_{i=1}^n(1-p_i+e^{ta_i}p_i)=\prod\limits_{i=1}^n(1+p_i(e^{ta_i}-1))\leq\prod\limits_{i=1}^n\exp(p_i(e^{ta_i}-1))$.
Since $0 \leq a_i \leq 1$, $\prod\limits_{i=1}^n\exp(p_i(e^{ta_i}-1))\leq\prod\limits_{i=1}^n\exp(p_i(e^t-1))=\exp(\sum\limits_{i=1}^np_i(e^t-1))=\exp(\mu(e^t-1))$.
Following the proof of Theorem 4.4 in the textbook, we get $\Pr(X\geq (1+\delta)\mu)=\Pr(e^{tX}\geq e^{t(1+\delta)\mu})\leq \frac{\exp((e^t-1)\mu)}{\exp(t(1+\delta)\mu)}$ for $t>0$.
Now take $t=\ln(1+\delta)$, and we show the desired Chernoff bound.\\
Similarly, $\Pr(X\leq (1-\delta)\mu)=\Pr(e^{tX}\geq e^{t(1-\delta)\mu})\leq\frac{\exp((e^t-1)\mu)}{\exp(t(1-\delta)\mu)}$ for $t<0$.
Now for $0<\delta<1$, take $t=\ln(1-\delta)$, and we can show that $\Pr(X\leq (1-\delta)\mu)\leq\left(\frac{e^{-\delta}}{(1-\delta)^{1-\delta}}\right)^\mu$.
\subsection{}
Note that $|(1-p_i)-(-p_i)|=1$, and $\textbf{E}[X_i]=0$ for all $i$.
Applying the Hoeffding bound to $X$, we get $\Pr(|\frac{1}{n}\sum\limits_{i=1}^nX_i | > \epsilon)=\Pr(|X|>n\epsilon)\leq2e^{-2n\epsilon^2}$.
Now take $\epsilon=\frac{a}{n}$ to get $\Pr(|X|>a)\leq 2e^{-2a^2/n}$.
\subsection{}
Let $Y_i=a_i(X_i-p_i)$. Then we observe that $\Pr(-a_ip_i \leq Y_i \leq a_i(1-p_i))=1$ and $\textbf{E}[Y_i]=0$.
Now applying the Hoeffding bound to $\sum\limits_{i=1}^nY_i$ as:\\
$\Pr(|\sum\limits_{i=1}^nY_i|\geq\delta\mu)\leq 2e^{-2n(\frac{\delta\mu}{n})^2\frac{1}{\max a_i^2}}\leq2e^{-2\delta^2\mu^2/n}$.
\subsection{}
Let the total time (in steps) of a single processor be $X=\sum\limits_{i=1}^{n/m}X_i$, where $X_i$ is the number of steps for the $i$th job of the processor.
Since $\textbf{E}[X_i]=p+(1-p)k$, we take $Y_i$ as $X_i=1+(k-1)Y_i$ so that $Y_i\sim Bernoulli(1-p)$. With $Y_i$, we get $X=(n/m)+(k-1)\sum\limits_{i=1}^{n/m}Y_i$.\\
Applying the Chernoff bounds to $Y=\sum\limits_{i=1}^{n/m}Y_i$, we get $\Pr(|Y-\textbf{E}[Y]|\geq \delta\textbf{E}[Y])\leq 2e^{-\textbf{E}[Y]\delta^2/3}$.
Since $\textbf{E}[Y]=(n/m)(1-p)$, we can bound $X$ as:\\
$\Pr(|X-(n/m)(1+(1-p)(k-1))|\geq\delta (n/m)(1-p)(k-1))\leq2e^{-\frac{n}{m}(1-p)\frac{\delta^2}{3}}$.
Using the union bound, we bound the total time $T$ to $\Pr(|T-(n/m)(1+(1-p)(k-1))|\geq\delta (n/m)(1-p)(k-1))\leq2me^{-\frac{n}{m}(1-p)\frac{\delta^2}{3}}$.
We can take $\delta=\sqrt{\frac{3\ln(2m/\epsilon)}{(n/m)(1-p)}}$ to derive the bound with probability of at least $1-\epsilon$.
\subsection{}
\newpage
\section{Balls, Bins, and Random Graphs}
\subsection{}
As $(1+1/n)^n$ increases, we find the smallest $n$ to reach the threshold.\\
$(1+1/n)^n$ first reaches $0.99e$ at $n=50$, and $0.999999e$ at $n=499982$.\\
Since $(1-1/n)^n$ also increases, we solve in a similar way. $(1-1/n)^n$ first reaches $0.99/e$ at $n=51$ and $0.999999/e$ at $n=499991$.
\subsection{}
Recall the formula used in the birthday paradox: If there are $N$ possibilities, then we solve for the smallest $n$ that satisfies
$\prod\limits_{i=1}^{n-1}(1-\frac{i}{N})\approx\prod\limits_{i=1}^{n-1}e^{-i/n}=e^{-(n-1)n/2N}<1/2$.
Note that we omitted the final approximation to derive exact numerical answers.\\
Regardless of whether the number of Social Security number digits is $9$ or $13$, using the last four digits gives $N=10000$ and this gives $n=119$.\\
In the case where the number of digits is $9$ ($N=10^9$), we get $n=37234$.\\
In the case where the number of digits is $13$ ($N=10^{13}$), we get $n=3723298$.
\subsection{}

\end{document}