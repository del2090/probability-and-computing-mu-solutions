\documentclass{article}
\usepackage{graphicx} % Required for inserting images
\usepackage{amsmath}
\usepackage{amssymb}

\title {Probability and Computing, 2nd Edition \\[2ex] \large Solutions to Chapter 4: Chernoff and Hoeffding Bounds}
\author{Hahndeul Kim}
\date{July 2025}

\begin{document}

\maketitle
\newpage
\section*{4.1}
Let the number of games that Alice wins be $X$, where $X\sim B(n,0.6)$. Alice will lose the tournament with probability $\Pr(X\leq \frac{n-1}{2})$.
Now, let $\delta$ s. t. $(1-\delta) \times \frac{3n}{5} = \frac{n-1}{2}$ to obtain the tightest bound.\\
$\Pr(X\leq \frac{n-1}{2})=\Pr(X\leq(1-\delta)\textbf{E}[X])\leq \exp(-\frac{3n}{5}\cdot \delta^2 \cdot \frac{1}{2})$\\
$=\exp(-\frac{1}{10}(\frac{1}{12}n+\frac{5}{6}+\frac{25}{12n}))\leq \exp(-\frac{1}{8})$ (AM-GM inequality).
\section*{4.2}
With Markov's inequality, $\Pr(X\geq n/4) \leq (n/6)/(n/4)=2/3$.\\
With Chebyshev's inequality, $\Pr(X\geq n/4) \leq \Pr(|X-n/6|\geq n/12)\leq \frac{\textbf{Var}[X]}{(n/12)^2}$\\
$=\frac{144}{n^2}\times(n\cdot \frac{1}{6}\cdot\frac{5}{6})=20/n$.\\
To use Chernoff bounds, let $\delta=1/2$. Then $\Pr(X\geq n/4)=\Pr(X\geq(1+\delta)\textbf{E}[X])$\\
$\leq\left(\frac{e^{0.5}}{1.5^{1.5}}\right)^{n/6}=\left(\frac{e}{1.5^3}\right)^{n/12}$.
\section*{4.3}
(a) Let $X\sim B(n,p)$. Then $M_X(t)=\textbf{E}[e^{tX}]=\sum\limits_{i=0}^ne^{it}\Pr(X=i)$\\
$=\sum\limits_{i=0}^ne^{it}\binom{n}{i}p^i(1-p)^{n-i}=\sum\limits_{i=0}^n\binom{n}{i}(pe^t)^i(1-p)^{n-i}=(pe^t+1-p)^n$.\\
(b) $M_{X+Y}(t)=\textbf{E}[e^{t(X+Y)}]=\textbf{E}[e^{tX}e^{tY}]=\textbf{E}[e^{tX}]\textbf{E}[e^{tY}]=(pe^t+1-p)^{m+n}$.\\
(c) Since moment generating function uniquely determines the distribution, $X+Y\sim B(m+n,p)$.
\section*{4.4}
Let the total number of heads be $X$, where $X\sim B(100,\frac{1}{2})$. Then we find $\Pr(X\geq55)\approx0.1841$.\\
From Chernoff bound, we find that $\Pr(X\geq (1+\frac{1}{10})50)\leq \exp(-\frac{50}{3}\cdot\frac{1}{10^2})=\exp(-\frac{1}{6})\approx0.8465$.\\
For $Y\sim B(1000,\frac{1}{2})$, $\Pr(Y\geq 550)\approx0.0009$.\\
From Chernoff bound, we find that $\Pr(Y\geq (1+\frac{1}{10})500)\leq \exp(-\frac{500}{3}\cdot\frac{1}{10^2})=\exp(-\frac{5}{3})\approx0.1889$.\\
\section*{4.5}
Let $Y=NX$, so that we aim to satisfy $\Pr(|Y-Np|>N\epsilon p) \leq \delta$. Consider that\\
$\Pr(Y>Np(1+\epsilon)) < \exp(-Np\cdot \frac{\epsilon^2}{3})$, and $\Pr(Y<Np(1-\epsilon)) < \exp(-Np\cdot \frac{\epsilon^2}{2})$.\\
Thus, we aim to satisfy $\exp(-Np\cdot \frac{\epsilon^2}{3})+\exp(-Np\cdot \frac{\epsilon^2}{2}) \leq 2\exp(-Np\cdot \frac{\epsilon^2}{3}) \leq \delta$.\\
$\therefore N \geq \frac{3}{p\epsilon^2} \ln \frac{2}{\delta}$. With $\epsilon=0.1$, $\delta=0.05$ and $0.2 \leq p \leq 0.8$, $N\geq1500\ln40\approx5533$.
\section*{4.6}
(a) Let $X\sim B(1000000, 0.02)$. Then $\Pr(X\geq 40000) \leq e^{-20000/3}$.\\
(b) Set $X$ and $Y$ as given and choose $k$, $l$ such that $l \leq k - 10000$ so that bounding $\Pr((X>k)\cap(Y<l))$ suffices.
As examples, we choose $k=15300$ and $l=4900$ here.
Since $X\sim B(510000, 0.02)$, $Y\sim B(490000,0.02)$ and $X \perp\!\!\!\perp Y$, $\Pr((X>k)\cap(Y<l)) = \Pr(X>k)\Pr(Y<l) \leq e^{-10200/12}\times e^{-9800/8} = e^{-2025}$.
\section*{4.7}
Recall that $M_X(t)=\prod\limits_{i=1}^n (p_ie^t + (1-p_i)) = \prod\limits_{i=1}^n (1+p_i(e^t-1))\leq \prod\limits_{i=1}^ne^{p_i(e^t-1)}$\\
$=e^{\mu(e^t-1)}$ holds when $X$ is the sum of Poisson trials ($\Pr(X_i=1)=p_i$).\\
Let $t=\ln (1+\delta)$ and follow the derivation of Chernoff bounds.\\
$\Pr(X\geq (1+\delta)\mu_H) \leq \frac{\textbf{E}[e^{tX}]}{e^{t(1+\delta)\mu_H}}\leq\frac{e^{\mu(e^t-1)}}{e^{t(1+\delta)\mu_H}}$
$\leq \left( \frac{e^{e^t-1}}{e^{t(1+\delta)}}\right)^{\mu_H}=\left( \frac{e^\delta}{(1+\delta)^{(1+\delta)}}\right)^{\mu_H}$.\\
Similarly, let $t=\ln (1-\delta)$ and prove the latter inequality.\\
$\Pr(X\leq (1-\delta)\mu_L) \leq \frac{\textbf{E}[e^{tX}]}{e^{t(1-\delta)\mu_L}}\leq\frac{e^{\mu(e^t-1)}}{e^{t(1-\delta)\mu_L}}$
$\leq \left( \frac{e^{e^t-1}}{e^{t(1-\delta)}}\right)^{\mu_L}=\left( \frac{e^{-\delta}}{(1-\delta)^{(1-\delta)}}\right)^{\mu_L}$. $\blacksquare$
\section*{4.8}
For any permutation $\pi$ produced with the given approach, $\Pr(f=\pi)=\prod\limits_{i=1}^n\frac{1}{k+1-i}$ holds.
Since the number of possible permutations is $\frac{k!}{(k-n)!}=\frac{1}{\Pr(f=\pi)}$,
the given approach produces a permutation chosen uniformly at random from all permutations.\\
Now, let $X_j$ be the number of black box calls to determine $f(j)$. Then $X_j\sim Geom(\frac{k+1-j}{k})$ holds.
Thus, $\textbf{E}[\sum\limits_{i=1}^nX_i]=\sum\limits_{i=1}^n\textbf{E}[X_i]=\sum\limits_{i=1}^n\frac{k}{k+1-i}$.\\
When $k=n$, $\textbf{E}[\sum\limits_{i=1}^nX_i]=\sum\limits_{i=1}^n\frac{n}{i}=nH(n)\approx n \ln n$.\\
Similarly, when $k=2n$, $\textbf{E}[\sum\limits_{i=1}^nX_i]=\sum\limits_{i=1}^n\frac{2n}{n+i}=2n(H(2n)-H(n))\approx2n\ln2$. In this case, $\frac{2n+1-j}{2n}\geq\frac{2n+1-n}{2n}\geq\frac{1}{2}$.\\
Now, to derive the desired Chernoff bound, we first compute the moment generating function of $X=\sum\limits_{i=1}^nX_j$.
Let $p_i=\frac{2n+1-i}{2n}$. Since $X_i$ are independent, $\textbf{E}[e^{tX}]=\prod\limits_{i=1}^n\textbf{E}[e^{tX_i}]$
$=\prod\limits_{i=1}^n\left(\prod\limits_{j=1}^\infty(e^{tj}p_i(1-p_i)^{j-1}\right)=\prod\limits_{i=1}^n\left(\frac{p_i}{1-p_i}\prod\limits_{j=1}^\infty(e^t(1-p_i))^j\right)$.\\
Suppose that we choose $t$ s. t. $0<t<\ln2$ when deriving the Chernoff bound.\\ Then $\textbf{E}[e^{tX}]=\prod\limits_{i=1}^n\frac{p_ie^t}{1-e^t(1-p_i)}$.
Since $t>0$, $\frac{\partial}{\partial p_i}\left(\frac{p_ie^t}{1-e^t(1-p_i)}\right)=\frac{1-e^t}{(1-e^t(1-p_i))^2}<0$.
This leads to $\textbf{E}[e^{tX}]=\prod\limits_{i=1}^n\frac{p_ie^t}{1-e^t(1-p_i)}\leq \left(\frac{\frac{1}{2}e^t}{1-\frac{1}{2}e^t}\right)^n$.\\
Now derive the desired Chernoff bound with $\Pr(X\geq 4n) \leq \frac{\textbf{E}[e^{tX}]}{e^{4nt}}\leq\left(\frac{1}{(2-e^t)e^{3t}}\right)^n$.\\
Since the function $(2-e^t)e^{3t}$ has its maximum at $t=\ln\frac{3}{2}$ and $0<\ln\frac{3}{2}<\ln2$, we choose $t=\ln\frac{3}{2}$ for the tightest possible bound.\\
The desired bound would be $\Pr(X\geq 4n) \leq \left(\frac{1}{(2-e^t)e^{3t}}\right)^n \Big|_{t=\ln\frac{3}{2}}=(\frac{16}{27})^n$.
\section*{4.9}
(a) By Chebyshev's inequality, $\Pr[|\sum\limits_{i=1}^tX_i-\textbf{E}[X]|\geq\epsilon\textbf{E}[X]]\leq\frac{\textbf{Var}[X]}{t(\epsilon\textbf{E}[X])^2}$
$=\frac{r^2}{t\epsilon^2}$. Thus, setting $t$ to satisfy $\frac{r^2}{t\epsilon^2}\leq \delta$ suffices.
This leads to $t \geq \frac{r^2}{\epsilon^2\delta}$, which proves the claim.\\
(b) Set $\delta=1-3/4=1/4$. Then we get $t \geq \frac{4r^2}{\epsilon^2}$, which proves the claim.\\
(c) Let $Y_i$ be indicator variables that are $1$ if $|X_i-\textbf{E}[X]|\geq\epsilon\textbf{E}[X]$. Then let the median of $Y_i$s be $m$, and bound the probability $\Pr(|m-\textbf{E}[X]|\geq\epsilon\textbf{E}[X])$.\\
Note that $\textbf{E}[\sum\limits_{i=1}^tY_i]\leq t/4$ by definition, and $|m-\textbf{E}[X]|\geq\epsilon\textbf{E}[X]$ holds only if $\sum\limits_{i=1}^tY_i \geq t/2$.
Then, $\Pr(|m-\textbf{E}[X]|\geq\epsilon\textbf{E}[X]) \leq \Pr\left(\sum\limits_{i=1}^tY_i \geq t/2\right)$. Let $Y=\sum\limits_{i=1}^tY_i$.
Then $\Pr(Y\geq t/2)=\Pr\left(Y\geq(1+(\frac{t}{2\textbf{E}[Y]}-1))\textbf{E}[Y]\right)$\\
$\leq \left(\frac{e^\delta}{(1+\delta)^{1+\delta}}\right)^{\textbf{E}[Y]}=(\frac{2e}{t})^{t/2}\times e^{-\textbf{E}[Y]}\textbf{E}[Y]^{t/2}$.\\
Since $\frac{\partial}{\partial \textbf{E}[Y]}\left( (\frac{2e}{t})^{t/2}e^{-\textbf{E}[Y]}\textbf{E}[Y]^{t/2} \right)=(\frac{2e}{t})^{t/2} e^{-\textbf{E}[Y]}\textbf{E}[Y]^{t/2-1}(t/2-\textbf{E}[Y])> 0$,\\
substitute $t/4$ for $\textbf{E}[Y]$ to derive our bound. Thus, $\Pr(Y\geq t/2) \leq (\frac{e}{4})^{t/4}$.\\
Here we need $t$ that satisfies $(\frac{e}{4})^{t/4} \leq \delta$, which leads to $t \geq \frac{4}{\ln \frac{4}{e}}\ln \frac{1}{\delta}$.
Therefore, together with 4.9.(b), we only need $O(\log(1/\delta))$ estimates constructed from $O(r^2\log(1/\delta)/\epsilon^2)$ samples.
\section*{4.10}
Let $X=\sum\limits_{i=1}^{1000000}X_i$ where $X_i$ denotes the winnings of the $i$th game.\\
Then by the Chernoff bound, $\Pr(X\geq10000) \leq \frac{\textbf{E}[e^{tX}]}{e^{10000t}}=\left(\frac{\textbf{E}[e^{tX_i}]^{100}}{e^{t}}\right)^{10000}$\\
$=\left(\frac{(167/200)e^{-t}+(4/25)e^{2t}+(1/200)e^{99t}}{e^{0.01t}}\right)^{1000000}$.
Using graph software, you can choose $t=0.0006$ and derive $\Pr(X\geq10000) \leq 0.0001606$.
\section*{4.11}
Since $\textbf{E}[X_i]=1$, $\textbf{E}[X]=n$. Thus, we bound $\Pr(X\geq (1+\delta)n)$ as\\
$\Pr(X\geq (1+\delta)n)\leq\frac{\textbf{E}[e^{tX}]}{e^{t(1+\delta)n}}$ with $t>0$. $\textbf{E}[e^{tX}]=\prod\limits_{i=1}^n\textbf{E}[e^{tX_i}]=\left(\frac{1}{3}(1+e^t+e^{2t})\right)^n$\\
leads to $\Pr(X\geq (1+\delta)n)\leq\left(\frac{1+e^t+e^{2t}}{3e^{t(1+\delta)}}\right)^n$.
Although $t=\frac{\delta+\sqrt{4-3\delta^2}}{1-\delta}$ minimizes $\frac{1+e^t+e^{2t}}{3e^{t(1+\delta)}}$, it is too complex to be used as a generalized bound.
Thus, we put $t=\ln(1+\delta)$ for simplicity and derive $\Pr(X\geq (1+\delta)n)\leq\left(\frac{3+3\delta+\delta^2}{3(1+\delta)^{(1+\delta)}}\right)^n$.\\
The Chernoff bound for $\Pr(X\leq(1-\delta)n)$ can also be derived in a similar way.
\section*{4.12}
(a) We can think of $X_i$ as the number of tails between $i-1$th head and $i$th head.
Now, let $Y_i$ be indicator variables that are $1$ if $i$th flip is head.
Then let $Y=\sum\limits_{i=1}^{(1+\delta)2n}Y_i$, and derive the Chernoff bound as $\Pr(X\geq(1+\delta)2n)=\Pr(Y\leq n)$.
Since $\textbf{E}[Y]=(1+\delta)n$, $\Pr(Y\leq n)=\Pr(Y\leq (1-\frac{\delta}{1+\delta})\textbf{E}[Y])\leq e^{-\frac{1}{2}\textbf{E}[Y](\frac{\delta}{1+\delta})^2}=e^{-\frac{n\delta^2}{2(1+\delta)}}$.\\
(b) Here, the moment generating function for $X$ can be derived as $\textbf{E}[e^{tX}]=\left(\frac{e^t}{2-e^t}\right)^n$ for $0 < t < \ln 2$ (refer to the solution for exercise 4.8).\\
Thus, $\Pr(X\geq(1+\delta)2n)\leq\frac{(\frac{e^t}{2-e^t})^n}{e^{t(1+\delta)2n}}=\left(\frac{1}{e^{t(1+2\delta)}(2-e^t)}\right)^n$.
Since $e^{t(1+2\delta)}(2-e^t)$ is maximized at $t=\ln(\frac{1+2\delta}{1+\delta}) < \ln 2$, we choose it to derive the tightest bound.
Therefore, $\Pr(X\geq(1+\delta)2n)\leq\left((\frac{1+\delta}{1+2\delta})^{1+2\delta}(1+\delta)\right)^n$.\\
(c) To compare two bounds, we inspect the sign of $e^{-\frac{\delta^2}{2(1+\delta)}}-(\frac{1+\delta}{1+2\delta})^{1+2\delta}(1+\delta)$.
The simpler equivalent would be $(1+2\delta)\ln(1+2\delta)-(2+2\delta)\ln(1+\delta)-\frac{\delta^2}{2(1+\delta)}$.\\
The computation can be performed numerically using $\ln(x+1)=\sum\limits_{n=1}^\infty\frac{(-1)^n}{n}x^n$, or with the help of graph software.
In either way, it can be shown that the bound derived in (b) is better.
\section*{4.13}
(a) From the Chernoff bound, $\Pr(X\geq xn) \leq \textbf{E}[e^{tX}]/e^{txn}$.\\
Since $\textbf{E}[e^{tX}]=\prod\limits_{i=1}^n\textbf{E}[e^{tX_i}]=(1-p+pe^t)^n$, $\Pr(X\geq xn) \leq \left( \frac{1-p+pe^t}{e^{xt}}\right)^n=((1-p)e^{-xt}+pe^{(1-x)t})^n$.
To derive the tightest bound, we solve for $\frac{\partial}{\partial t} ((1-p)e^{-xt}+pe^{(1-x)t})=0$, which gives $t=\ln(x(1-p))-\ln((1-x)p)$.
Since $(1-p)e^{-xt}+pe^{(1-x)t}$ is convex w. r. t. $t$ with given conditions, this gives the minimum.
By plugging this in, we can show that $\Pr(X\geq xn) \leq e^{-nF(x,p)}$.\\
(b) Since $\frac{\partial^2}{\partial x^2} (F(x,p)-2(x-p)^2)=\frac{1}{x}+\frac{1}{1-x}-4=\frac{(2x-1)^2}{x(1-x)}\geq 0$, $F(x,p)-2(x-p)^2$ is convex w. r. t. $x$ when $0<x,p<1$.
Considering that $\frac{\partial}{\partial x}(F(x,p)-2(x-p)^2)=0$ yields $x=p$ and $(F(x,p)-2(x-p)^2) \Big|_{x=p}=0$, we get $F(x,p)-2(x-p)^2 \geq 0$.\\
(c) $\Pr(X\geq(p+\epsilon)n)\leq e^{-nF(p+\epsilon,p)}$ holds by (a), and $e^{-nF(p+\epsilon,p)} \leq e^{-n\times 2(p+\epsilon-p)^2}=e^{-2n\epsilon^2}$ holds by (b).\\
(d) Take $Y_i=1-X_i$, and let $Y=n-X$. Then, $\Pr(X\leq (p-\epsilon)n)=\Pr(Y\geq((1-p)+\epsilon)n)\leq e^{-2n\epsilon^2}$ holds by (c).
Combined with (c), we get $\Pr(|X-pn|\geq \epsilon n)=\Pr(X\leq(p-\epsilon)n)+\Pr(X\geq(p+\epsilon)n)\leq 2e^{-2n\epsilon^2}$.
\section*{4.14}
We first bound the moment generating function of $X$ to $\textbf{E}[e^{tX}]=\prod\limits_{i=1}^n(1-p_i+e^{ta_i}p_i)=\prod\limits_{i=1}^n(1+p_i(e^{ta_i}-1))\leq\prod\limits_{i=1}^n\exp(p_i(e^{ta_i}-1))$.
Since $0 \leq a_i \leq 1$, $\prod\limits_{i=1}^n\exp(p_i(e^{ta_i}-1))\leq\prod\limits_{i=1}^n\exp(p_i(e^t-1))=\exp(\sum\limits_{i=1}^np_i(e^t-1))=\exp(\mu(e^t-1))$.
Following the proof of Theorem 4.4 in the textbook, we get $\Pr(X\geq (1+\delta)\mu)=\Pr(e^{tX}\geq e^{t(1+\delta)\mu})\leq \frac{\exp((e^t-1)\mu)}{\exp(t(1+\delta)\mu)}$ for $t>0$.
Now take $t=\ln(1+\delta)$, and we show the desired Chernoff bound.\\
Similarly, $\Pr(X\leq (1-\delta)\mu)=\Pr(e^{tX}\geq e^{t(1-\delta)\mu})\leq\frac{\exp((e^t-1)\mu)}{\exp(t(1-\delta)\mu)}$ for $t<0$.
Now for $0<\delta<1$, take $t=\ln(1-\delta)$, and we can show that $\Pr(X\leq (1-\delta)\mu)\leq\left(\frac{e^{-\delta}}{(1-\delta)^{1-\delta}}\right)^\mu$.
\section*{4.15}
Note that $|(1-p_i)-(-p_i)|=1$, and $\textbf{E}[X_i]=0$ for all $i$.
Applying the Hoeffding bound to $X$, we get $\Pr(|\frac{1}{n}\sum\limits_{i=1}^nX_i | > \epsilon)=\Pr(|X|>n\epsilon)\leq2e^{-2n\epsilon^2}$.
Now take $\epsilon=\frac{a}{n}$ to get $\Pr(|X|>a)\leq 2e^{-2a^2/n}$.
\section*{4.16}
Let $Y_i=a_i(X_i-p_i)$. Then we observe that $\Pr(-a_ip_i \leq Y_i \leq a_i(1-p_i))=1$ and $\textbf{E}[Y_i]=0$.
Now applying the Hoeffding bound to $\sum\limits_{i=1}^nY_i$ as:\\
$\Pr(|\sum\limits_{i=1}^nY_i|\geq\delta\mu)\leq 2e^{-2n(\frac{\delta\mu}{n})^2\frac{1}{\max a_i^2}}\leq2e^{-2\delta^2\mu^2/n}$.
\section*{4.17}
Let the total time (in steps) of a single processor be $X=\sum\limits_{i=1}^{n/m}X_i$, where $X_i$ is the number of steps for the $i$th job of the processor.
Since $\textbf{E}[X_i]=p+(1-p)k$, we take $Y_i$ as $X_i=1+(k-1)Y_i$ so that $Y_i\sim Bernoulli(1-p)$. With $Y_i$, we get $X=(n/m)+(k-1)\sum\limits_{i=1}^{n/m}Y_i$.\\
Applying the Chernoff bounds to $Y=\sum\limits_{i=1}^{n/m}Y_i$, we get $\Pr(|Y-\textbf{E}[Y]|\geq \delta\textbf{E}[Y])\leq 2e^{-\textbf{E}[Y]\delta^2/3}$.
Since $\textbf{E}[Y]=(n/m)(1-p)$, we can bound $X$ as:\\
$\Pr(|X-(n/m)(1+(1-p)(k-1))|\geq\delta (n/m)(1-p)(k-1))\leq2e^{-\frac{n}{m}(1-p)\frac{\delta^2}{3}}$.
Using the union bound, we bound the total time $T$ to $\Pr(|T-(n/m)(1+(1-p)(k-1))|\geq\delta (n/m)(1-p)(k-1))\leq2me^{-\frac{n}{m}(1-p)\frac{\delta^2}{3}}$.
We can take $\delta=\sqrt{\frac{3\ln(2m/\epsilon)}{(n/m)(1-p)}}$ to derive the bound with probability of at least $1-\epsilon$.
\section*{4.18}

\end{document}
