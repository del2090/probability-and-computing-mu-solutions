\documentclass{article}
\usepackage{graphicx} % Required for inserting images
\usepackage{amsmath}
\usepackage{amssymb}

\title {Probability and Computing, 2nd Edition \\[2ex] \large Solutions to Chapter 4: Chernoff and Hoeffding Bounds}
\author{Hahndeul Kim}
\date{July 2025}

\begin{document}

\maketitle
\newpage
\section*{4.1}
Let the number of games that Alice wins be $X$, where $X\sim B(n,0.6)$. Alice will lose the tournament with probability $\Pr(X\leq \frac{n-1}{2})$.
Now, let $\delta$ s. t. $(1-\delta) \times \frac{3n}{5} = \frac{n-1}{2}$ to obtain the tightest bound.\\
$\Pr(X\leq \frac{n-1}{2})=\Pr(X\leq(1-\delta)\textbf{E}[X])\leq \exp(-\frac{3n}{5}\cdot \delta^2 \cdot \frac{1}{2})$\\
$=\exp(-\frac{1}{10}(\frac{1}{12}n+\frac{5}{6}+\frac{25}{12n}))\leq \exp(-\frac{1}{8})$ (AM-GM inequality).
\section*{4.2}
With Markov's inequality, $\Pr(X\geq n/4) \leq (n/6)/(n/4)=2/3$.\\
With Chebyshev's inequality, $\Pr(X\geq n/4) \leq \Pr(|X-n/6|\geq n/12)\leq \frac{\textbf{Var}[X]}{(n/12)^2}$\\
$=\frac{144}{n^2}\times(n\cdot \frac{1}{6}\cdot\frac{5}{6})=20/n$.\\
To use Chernoff bounds, let $\delta=1/2$. Then $\Pr(X\geq n/4)=\Pr(X\geq(1+\delta)\textbf{E}[X])$\\
$\leq\left(\frac{e^{0.5}}{1.5^{1.5}}\right)^{n/6}=\left(\frac{e}{1.5^3}\right)^{n/12}$.
\section*{4.3}
(a) Let $X\sim B(n,p)$. Then $M_X(t)=\textbf{E}[e^{tX}]=\sum\limits_{i=0}^ne^{it}\Pr(X=i)$\\
$=\sum\limits_{i=0}^ne^{it}\binom{n}{i}p^i(1-p)^{n-i}=\sum\limits_{i=0}^n\binom{n}{i}(pe^t)^i(1-p)^{n-i}=(pe^t+1-p)^n$.\\
(b) $M_{X+Y}(t)=\textbf{E}[e^{t(X+Y)}]=\textbf{E}[e^{tX}e^{tY}]=\textbf{E}[e^{tX}]\textbf{E}[e^{tY}]=(pe^t+1-p)^{m+n}$.\\
(c) Since moment generating function uniquely determines the distribution, $X+Y\sim B(m+n,p)$.
\section*{4.4}
Let the total number of heads be $X$, where $X\sim B(100,\frac{1}{2})$. Then we find $\Pr(X\geq55)\approx0.1841$.\\
From Chernoff bound, we find that $\Pr(X\geq (1+\frac{1}{10})50)\leq \exp(-\frac{50}{3}\cdot\frac{1}{10^2})=\exp(-\frac{1}{6})\approx0.8465$.\\
For $Y\sim B(1000,\frac{1}{2})$, $\Pr(Y\geq 550)\approx0.0009$.\\
From Chernoff bound, we find that $\Pr(Y\geq (1+\frac{1}{10})500)\leq \exp(-\frac{500}{3}\cdot\frac{1}{10^2})=\exp(-\frac{5}{3})\approx0.1889$.\\
\section*{4.5}
Let $Y=NX$, so that we aim to satisfy $\Pr(|Y-Np|>N\epsilon p) \leq \delta$. Consider that\\
$\Pr(Y>Np(1+\epsilon)) < \exp(-Np\cdot \frac{\epsilon^2}{3})$, and $\Pr(Y<Np(1-\epsilon)) < \exp(-Np\cdot \frac{\epsilon^2}{2})$.\\
Thus, we aim to satisfy $\exp(-Np\cdot \frac{\epsilon^2}{3})+\exp(-Np\cdot \frac{\epsilon^2}{2}) \leq 2\exp(-Np\cdot \frac{\epsilon^2}{3}) \leq \delta$.\\
$\therefore N \geq \frac{3}{p\epsilon^2} \ln \frac{2}{\delta}$. With $\epsilon=0.1$, $\delta=0.05$ and $0.2 \leq p \leq 0.8$, $N\geq1500\ln40\approx5533$.
\section*{4.6}
(a) Let $X\sim B(1000000, 0.02)$. Then $\Pr(X\geq 40000) \leq e^{-20000/3}$.\\
(b) Set $X$ and $Y$ as given and choose $k$, $l$ such that $l \leq k - 10000$ so that bounding $\Pr((X>k)\cap(Y<l))$ suffices.
As examples, we choose $k=15300$ and $l=4900$ here.
Since $X\sim B(510000, 0.02)$, $Y\sim B(490000,0.02)$ and $X \perp\!\!\!\perp Y$, $\Pr((X>k)\cap(Y<l)) = \Pr(X>k)\Pr(Y<l) \leq e^{-10200/12}\times e^{-9800/8} = e^{-2025}$.
\section*{4.7}

\end{document}
