\documentclass{article}
\usepackage{graphicx} % Required for inserting images
\usepackage{amsmath}
\usepackage{amssymb}

\title {Probability and Computing, 2nd Edition \\[2ex] \large Solutions to Chapter 5: Balls, Bins, and Random Graphs}
\author{Hahndeul Kim}
\date{July 2025}

\begin{document}

\maketitle
\newpage
\section*{5.1}
As $(1+1/n)^n$ increases, we find the smallest $n$ to reach the threshold.\\
$(1+1/n)^n$ first reaches $0.99e$ at $n=50$, and $0.999999e$ at $n=499982$.\\
Since $(1-1/n)^n$ also increases, we solve in a similar way. $(1-1/n)^n$ first reaches $0.99/e$ at $n=51$ and $0.999999/e$ at $n=499991$.
\section*{5.2}
Recall the formula used in the birthday paradox: If there are $N$ possibilities, then we solve for the smallest $n$ that satisfies
$\prod\limits_{i=1}^{n-1}(1-\frac{i}{N})\approx\prod\limits_{i=1}^{n-1}e^{-i/n}=e^{-(n-1)n/2N}<1/2$.
Note that we omitted the final approximation to derive exact numerical answers.\\
Regardless of whether the number of Social Security number digits is $9$ or $13$, using the last four digits gives $N=10000$ and this gives $n=119$.\\
In the case where the number of digits is $9$ ($N=10^9$), we get $n=37234$.\\
In the case where the number of digits is $13$ ($N=10^{13}$), we get $n=3723298$.
\section*{5.3}
Let the number of balls thrown be $m$. Then the desired probability is $\prod\limits_{i=0}^{m-1}(1-\frac{i}{n})$.\\
We first determine $c_1$.
$m=c_1\sqrt{n}$ should satisfy $\prod\limits_{i=0}^{m-1}(1-\frac{i}{n}) \leq \prod\limits_{i=0}^{m-1}e^{-i/n}=e^{-(m-1)m/2n}\leq e^{-1}$.
Since $(m-1)m=c_1^2n-c_1\sqrt{n}\geq 2n$, $(c_1^2-2)\sqrt{n}\geq c_1$.
Therefore, we choose $c_1$ that is greater than or equal to $\frac{1}{2}\left(\frac{1}{\sqrt{n}}+\sqrt{\frac{1}{n}+8}\right)$.\\
Now we determine $c_2$. To use the given hint, assume that $2m < n$.\\
$\prod\limits_{i=0}^{m-1}(1-\frac{i}{n})\geq\prod\limits_{i=0}^{m-1}\exp(-\frac{i}{n}-\frac{i^2}{n^2})=\exp(-\frac{m(m-1)}{2n}-\frac{(m-1)m(2m-1)}{6n^2})$\\
$=\exp(-\frac{m(m-1)}{2n}(1+\frac{2m-1}{3n}))\geq\exp(-\frac{m^2}{2n}(1+\frac{2m}{3n}))\geq \frac{1}{2}$ should be satisfied for $m=c_2\sqrt{n}$.
This is equivalent to satisfying $\frac{c_2^2}{2}(1+\frac{2c_2}{3\sqrt{n}})\leq \ln 2$.\\
Since $n$ is sufficiently large, choosing $c_2 = \sqrt{2\ln2 - \frac{1}{\ln n}}$ yields the desired result.
\section*{5.4}
Let event $A$ indicate that there exist two or more people who share a birthday, and event $B$ indicate that exactly two people share a birthday.
Then our desired probability would be $\Pr(A-B)=\Pr(A)-\Pr(B)$ since $B\subset A$.\\
We first determine $\Pr(A)$, which is easy: $\Pr(A)=1-\prod\limits_{i=1}^{100}\frac{366-i}{365}$.\\
We now determine $\Pr(B)$. If there are $i$ shared birthdays in which each day is shared by exactly two people, then the number of possible permutations would be the product of the following terms:\\
$\binom{365}{i}$ ways to choose $i$ shared days, 
$\binom{100}{2i}$ ways to choose $2i$ people to share birthdays,
$\prod\limits_{j=1}^i\binom{2j}{2}$ ways to distribute $i$ birthdays to $2i$ people
and $\prod\limits_{j=1}^{100-2i}(366-i-j)$ ways to distribute unique birthdays to the rest.\\
Thus, $\Pr(B)=\sum\limits_{i=0}^{50}\frac{365!100!}{i!(100-2i)!(265+i)!2^i}\times\frac{1}{365^{100}}$.\\
Therefore, we can determine our desired probability $\Pr(A-B)=\Pr(A)-\Pr(B)$.
\section*{5.5}
Let $X\sim Poisson(\lambda)$. Then $M_X(t)=\textbf{E}[e^{tX}]=e^{\lambda(e^t-1)}$ holds.
By computing the second derivative of $M_X(t)$ with respect to $t$ and plugging $t=0$ in, we get $\textbf{E}[X^2]=\lambda+\lambda^2$.
Thus, $\textbf{Var}[X]=\lambda$ follows.
\section*{5.6}
We first show that $Y\sim Poisson(\mu p)$.\\
$\Pr(Y=k)=\sum\limits_{i=k}^\infty\Pr(X=i)\binom{i}{k}p^k(1-p)^{i-k}=\sum\limits_{i=k}^\infty\frac{e^{-\mu}\mu^i}{i!}\frac{i!}{k!(i-k)!}p^k(1-p)^{i-k}$\\
$=\frac{e^{-\mu}(p\mu)^k}{k!}\sum\limits_{i=k}^\infty\frac{(\mu(1-p))^{i-k}}{(i-k)!}=\frac{e^{-\mu}(p\mu)^k}{k!}e^{\mu(1-p)}=\frac{e^{-\mu p}(\mu p)^k}{k!}$.\\
We can also similarly show that $Z\sim Poisson(\mu(1-p))$.\\
Now we show that $\Pr(Y=i,Z=j)=\Pr(Y=i)\Pr(Z=j)$. Note that $X=Y+Z$ by definition.
This allows us to write $\Pr(Y=i,Z=j)$ as $\Pr(Y=i,X=i+j)=\Pr(X=i+j)\binom{i+j}{i}p^{i}(1-p)^{j}=\frac{e^{-\mu}\mu^{i+j}}{i!j!}p^i(1-p)^j$.\\
Since $\Pr(Y=i)\Pr(Z=j)=\frac{e^{-p\mu}(p\mu)^i}{i!}\frac{e^{-(1-p)\mu}((1-p)\mu)^j}{j!}=\frac{e^{-\mu}\mu^{i+j}p^i(1-p)^j}{i!j!}=\Pr(Y=i,X=i+j)$,
$Y \perp\!\!\!\perp Z$. $\blacksquare$
\section*{5.7}
We first prove that $\ln(1+x)\leq x$, which is equivalent to $1+x\leq e^x$.\\
Since $\ln(1+x)-x= -\frac{x^2}{2}+\frac{x^3}{3}-\cdots$, this can be seen as an alternating series as $\frac{x^n}{n}$ is monotonically decreasing in $|x| \leq 1$.
We can apply rearrangements to the alternating series as $\ln(1+x)-x=-\sum\limits_{n=1}^\infty\left(\frac{x^{2n}}{2n}-\frac{x^{2n+1}}{2n+1}\right)$,
since the Taylor expansion of $\ln(1+x)$ is absolutely convergent (to $e^x-1$).
The rearrangement gives $\ln(1+x)-x \leq 0$, which is the desired result.\\
We now prove $x + \ln(1-x^2) \leq \ln(1+x)$, which is equivalent to $e^x(1-x^2) \leq 1+x$.\\
Since $\ln(1-x^2)=\ln(1+x)+\ln(1-x)$, $x + \ln(1-x^2) \leq \ln(1+x)$ is reduced to $\ln(1-x)\leq -x$. 
At $|x|\leq 1$, this is equivalent to $\ln(1+x)\leq x$, which we have previously proved. $\blacksquare$
\section*{5.8}
(a) Since the ball is equally likely to fall in one of the three bins, the desired probability is $1/3$.\\
(b) Since the bin $2$ did not receive balls, we can simply think of this as throwing balls $n$ into $n-1$ bins. The conditional expectation would be $n/(n-1)$.\\
(c) Note that the probability that bin $1$ receives more balls than bin $2$ is the same as that of bin $2$ receiving more balls than bin $1$. Thus, we first compute the probability that two bins receive the same number of balls, which is\\
$P=\sum\limits_{k=0}^{\lfloor n/2\rfloor}\binom{n}{2k}\binom{2k}{k}(\frac{1}{n})^{2k}(1-\frac{2}{n})^{n-2k}$.
The desired probability would be $(1-P)/2$.
\section*{5.9}
In the given condition, the expected number of elements in a single bucket is at most $a$. Since $a=O(1)$, sorting all buckets can still be done in linear time.
\section*{5.10}
(a) By the Poisson approximation, the probability $p$ is bounded as $p \leq e\sqrt{n}(\frac{1}{e})^n$.\\
(b) $\frac{n!}{n^n}$.\\
(c) Since $\Pr(Z=n)=\frac{e^{-n}n^n}{n!}$ when $Z\sim Poisson(n)$, $\frac{n!}{n^n}\times\frac{e^{-n}n^n}{n!}=e^{-n}$ shows the claim.
Theorem $5.6$ states that the distribution $(Y_1,...,Y_n)$ constrained on $\sum\limits_{i=1}^n Y_i=n$ is equivalent to the balls and bins model.
Note that each $Y_i$ follows $Poisson(1)$ and each $X_i$ denotes the load of the $i$th bin in the balls and bins model.
Then using theorem $5.6$, $(1/e)^n/(\frac{e^{-n}n^n}{n!})=\frac{\Pr(\forall i, Y_i=1)}{\Pr(\sum_iY_i=n)}=\Pr(\forall i,Y_i=1 | \sum_i Y_i=n)=\Pr(\forall i, X_i=1)=\frac{n!}{n^n}$.
\section*{5.11}
Let $X_i$ be the indicator variables that are $1$ if there is a $k$-gap starting at $i$.\\
Let $X=\sum\limits_{i=0}^{n-k}X_i$.\\
(a) The expected number of $k$-gaps would be $\sum\limits_{i=0}^{n-k}\textbf{E}[X_i]=(n-k+1)(1-\frac{k}{n})^m$.\\
(b) First, we assume that the bin loads follow the Poisson distribution to derive the Poisson-approximated Chernoff bound.
We divide $\{X_i\}$ into $k$ subsets, so that all indicator variables in the same subset are independent.
First, we derive the Chernoff bound for $Y_0=\sum\limits_{i\geq0}X_{ik}$ where $0<\delta<1$.\\
$\Pr(Y_0\geq(1+\delta)\textbf{E}[Y_0])\leq\exp(-\frac{1}{3}\frac{\textbf{E}[X]}{k}\delta^2)$, and
$\Pr(Y_0\leq(1-\delta)\textbf{E}[Y_0])\leq\exp(-\frac{1}{2}\frac{\textbf{E}[X]}{k}\delta^2)$ holds.
Therefore, $\Pr(|Y_0-\textbf{E}[Y_0]|\geq\delta\textbf{E}[Y_0])\leq2\exp(-\frac{1}{3}\frac{\textbf{E}[X]}{k}\delta^2)$.\\
With the union bound for all $Y_i$, we get $\Pr(|X-\textbf{E}[X]|\geq\delta\textbf{E}[X])\leq2k\exp(-\frac{1}{3}\frac{\textbf{E}[X]}{k}\delta^2)$.
Since we have used the Poisson approximation to compute the Chernoff bound, the computed upper bound should be multiplied by $e\sqrt{m}$.
\section*{5.12}
Let $X_i$ be the indicator variables that are $1$ if a ball landed in bin $i$ by itself.\\
(a) The expected number of balls to be $\textit{served}$ in this round would be
$\sum\limits_{i=1}^n\textbf{E}[X_i]=\sum\limits_{i=1}^nb\times\frac{1}{n}(1-\frac{1}{n})^{b-1}=b(1-\frac{1}{n})^{b-1}$.
Therefore, the expected number of balls at the start of the next round would be $b(1-(1-\frac{1}{n})^{b-1})$.\\
(b) Note that if $n=1$, then the number of rounds required would be trivially $1$. Therefore, we only consider the cases where $n\geq 2$.\\
Since $x_{j+1}=x_j(1-(1-\frac{1}{n})^{x_j-1})\leq x_j(1-(1-\frac{x_j-1}{n}))=x_j\frac{x_j-1}{n}\leq \frac{x_j^2}{n}$, the inequality given in the hint is true.\\
With $x_1=n(1-(1-\frac{1}{n})^{n-1})$, cascading the inequality yields $x_k\leq n(\frac{x_1}{n})^{2^{k-1}}=n(1-(1-\frac{1}{n})^{n-1})^{2^{k-1}}\leq n(1-(1-\frac{1}{n})^n)^{2^{k-1}}$.\\
Now, let $k^*$ be the minimum $k$ that satisfies $n(1-(1-\frac{1}{n})^n)^{2^{k-1}}\leq 1$, so that the operation ends after no more than $k^*+1$ rounds.
Since $1-(1-\frac{1}{n})^n \leq \frac{3}{4}$ at $n\geq 2$, we calculate the minimum $k$ that satisfies $n(\frac{3}{4})^{2^{k-1}}\leq 1$.\\
Taking log on both sides gives $\ln n +2^{k^*-1} \ln\frac{3}{4}\leq 0$, which is equivalent to $2^{k^*-1}\geq \frac{\ln n}{\ln \frac{4}{3}}$.
Therefore, we get $k^*-1\geq \ln \frac{\ln n}{\ln \frac{4}{3}}$, which shows that $k^*=O(\log\log n)$.
\section*{5.13}
Let the load of bin $i$ be $X_i$, and let $Y_k=X_{kn/\log_2n}$ where $k\in\mathbb{N}_0$.
Then for all $i$, there exists $k\in \mathbb{N}$ such that $kn/\log_2 n \leq i \leq (k+1)n/\log_2 n$.\\
When a ball is thrown into the bin $i$, only one of the two bins (bin $kn/\log_2 n$ and bin $(k+1)n/\log_2 n$) must be chosen together. This means that $X_i \leq Y_k+Y_{k+1}$.\\
Since only one of the bins in $S=\{i|i=kn/\log_2 n\}$ gets a ball within a player's round, we can see $Y_k$ as each bin in the model of balls and bins with $\log_2 n$ bins and $\log_2 n$ balls.\\
Recall that the probability that the maximum load is more than $3\ln n/\ln\ln n$ is at most $1/n$ in the model of $n$ balls and $n$ bins (Lemma 5.1).
Therefore, we bound the probability that the maximum load is greater than $6M=6\ln \log_2 n / \ln\ln \log_2 n$, considering $X_i \leq Y_k + Y_{k+1}$.
Thus, we can derive the upper bound as $\Pr(\sum\limits_{i=0}^{n-1}\textbf{1}_{X_i\geq6M}>0)\leq\Pr(\sum\limits_{i=0}^{\log_2 n-1}\textbf{1}_{Y_i\geq 3M}>0)\leq \frac{1}{\log_2 n}$.
Thus, the maximum load is less than $6\ln \log_2 n / \ln\ln \log_2 n = O(\log\log n/\log\log\log n)$ with probability $1-\frac{1}{\log_2 n}$ which approaches $1$ as $n\rightarrow \infty$.
\section*{5.14}
(a) $\Pr(Z=\mu+h)-\Pr(Z=\mu-h-1)=\frac{e^{-\mu}\mu^{\mu+h}}{(\mu+h)!}-\frac{e^{-\mu}\mu^{\mu-h-1}}{(\mu-h-1)!}$
$=\frac{\mu^{\mu-h}}{(\mu+h)!}(\mu^{2h}-\sum\limits_{i=1}^h(\mu^2-i^2))$.
Since $\mu^2 > \mu^2 - i^2$, $\Pr(Z=\mu+h)-\Pr(Z=\mu-h-1)\geq 0$ holds and the claim is proved.\\
(b) $\Pr(Z\geq \mu)\geq \sum\limits_{h=0}^{\mu-1}\Pr(Z=\mu+h)\geq\sum\limits_{h=0}^{\mu-1}\Pr(Z=\mu-h-1)=\Pr(Z<\mu)=1-\Pr(Z\geq\mu)$
shows that $2\Pr(Z\geq\mu)\geq1$, proving the claim.\\
(c) Numerical validation can show that $\Pr(Z\geq \mu)\leq 1/2$ for all integers $\mu$ from $1$ to $10$.
\section*{5.15}
(a) $\textbf{E}[f(Y_1^{(m)},...,Y_n^{(m)})]=\sum\limits_{k=0}^\infty\textbf{E}[f(X_1^{(k)},...,X_n^{(k)})]\Pr(\sum\limits_{i=1}^nY_i^{(m)}=k)$
holds (recall the proof of Theorem 5.7).\\
If $\mu(m)=\textbf{E}[f(X_1^{(m)},...,X_n^{(m)})]$ is monotonically increasing in $m$, then we have $\mu(k)\geq\mu(m)$ for $k\geq m$ for some $m$.
Thus, $\textbf{E}[f(Y_1^{(m)},...,Y_n^{(m)})]=\sum\limits_{k<m}\mu(k)\Pr(\sum\limits_{i=1}^n Y_i^{(m)}=k)+\sum\limits_{k\geq m}\mu(k)\Pr(\sum\limits_{i=1}^n Y_i^{(m)}=k)$ ($f\geq 0$)\\
$\geq \sum\limits_{k\geq m}\mu(m)\Pr(\sum\limits_{i=1}^n Y_i^{(m)}=k)=\textbf{E}[f(X_1^{(m)},...,X_n^{(m)})]\Pr(\sum Y_i^{(m)}\geq m)$.\\
Similarly, if $\mu(m)$ is monotonically decreasing in $m$, then we have $\mu(k)\geq\mu(m)$ for $k\leq m$ for some $m$. Thus,\\
$\textbf{E}[f(Y_1^{(m)},...,Y_n^{(m)})]=\sum\limits_{k\leq m}\mu(k)\Pr(\sum\limits_{i=1}^n Y_i^{(m)}=k)+\sum\limits_{k> m}\mu(k)\Pr(\sum\limits_{i=1}^n Y_i^{(m)}=k)$\\
$\geq \sum\limits_{k\leq m}\mu(m)\Pr(\sum\limits_{i=1}^n Y_i^{(m)}=k)=\textbf{E}[f(X_1^{(m)},...,X_n^{(m)})]\Pr(\sum Y_i^{(m)}\leq m)$.\\
(b) Since the sum of independent Poisson random variables is also a Poisson random variable, $\sum Y_i^{(m)}\sim Poisson(m)$.
Using the result of exercise 5.14.(b) and 5.15.(a), $\textbf{E}[f(Y_1^{(m)},...,Y_n^{(m)})]\geq \textbf{E}[f(X_1^{(m)},...,X_n^{(m)})]\Pr(\sum Y_i^{(m)}\geq m)$\\
$\geq \textbf{E}[f(X_1^{(m)},...,X_n^{(m)})]\times \frac{1}{2}$.
Thus, Theorem 5.10 is proved for the monotonically increasing case.\\
If one can derive a formal proof on the statement of exercise 5.14.(c), Theorem 5.10 can also be proved for the monotonically decreasing case.
\section*{5.16}

\end{document}