\documentclass{article}
\usepackage{graphicx} % Required for inserting images
\usepackage{amsmath}
\usepackage{amssymb}

\title {Probability and Computing, 2nd Edition \\[2ex] \large Solutions to Chapter 5: Balls, Bins, and Random Graphs}
\author{Hahndeul Kim}
\date{July 2025}

\begin{document}

\maketitle
\newpage
\section*{5.1}
As $(1+1/n)^n$ increases, we find the smallest $n$ to reach the threshold.\\
$(1+1/n)^n$ first reaches $0.99e$ at $n=50$, and $0.999999e$ at $n=499982$.\\
Since $(1-1/n)^n$ also increases, we solve in a similar way. $(1-1/n)^n$ first reaches $0.99/e$ at $n=51$ and $0.999999/e$ at $n=499991$.
\section*{5.2}
Recall the formula used in the birthday paradox: If there are $N$ possibilities, then we solve for the smallest $n$ that satisfies
$\prod\limits_{i=1}^{n-1}(1-\frac{i}{N})\approx\prod\limits_{i=1}^{n-1}e^{-i/n}=e^{-(n-1)n/2N}<1/2$.
Note that we omitted the final approximation to derive exact numerical answers.\\
Regardless of whether the number of Social Security number digits is $9$ or $13$, using the last four digits gives $N=10000$ and this gives $n=119$.\\
In the case where the number of digits is $9$ ($N=10^9$), we get $n=37234$.\\
In the case where the number of digits is $13$ ($N=10^{13}$), we get $n=3723298$.
\section*{5.3}

\end{document}