\documentclass{article}
\usepackage{graphicx} % Required for inserting images
\usepackage{amsmath}
\usepackage{amssymb}

\title {Probability and Computing, 2nd Edition \\[2ex] \large Chapter 3 Solutions}
\author{Hahndeul Kim}
\date{June 2025}

\begin{document}

\maketitle
\newpage
\section*{3.1}
$\textbf{E}[X^2]=\sum\limits_{i=1}^n\frac{1}{n}\times i^2 = \frac{(n+1)(2n+1)}{6}$, and $\textbf{E}[X]=\frac{n+1}{2}$.\\
Thus, $\textbf{Var}[X]=\textbf{E}[X^2]-(\textbf{E}[X])^2=\frac{n^2-1}{12}$.
\section*{3.2}
$\textbf{E}[X]=0$, and $\textbf{E}[X^2]=\sum\limits_{i=1}^k \frac{2}{2k+1} \times i^2 = \frac{k(k+1)}{3}$.\\
Thus, $\textbf{Var}[X]=\textbf{E}[X^2]-(\textbf{E}[X])^2=\frac{k(k+1)}{3}$.
\section*{3.3}
The variance of a single die roll is $\frac{35}{12}$ from problem 3.1.
Since all rolls are independent, $\Pr(|X-350| \geq 50) \leq \frac{1}{50^2}\times \frac{35}{12} \times 100 = \frac{7}{60}$.
\section*{3.4}
$\textbf{Var}[cX]=\textbf{E}[(cX-\textbf{E}[cX])^2]=\textbf{E}[c^2X^2 - 2cX\textbf{E}[cX]+(\textbf{E}[cX])^2]$\\
$=c^2(\textbf{E}[X^2] - (\textbf{E}[X])^2)=c^2\textbf{Var}[X]$. $\blacksquare$
\section*{3.5}
$\textbf{Var}[X-Y]=\textbf{E}[((X-Y)-\textbf{E}[X-Y])^2]=\textbf{E}[((X-\textbf{E}[X])-(Y-\textbf{E}[Y]))^2]$\\
$=\textbf{E}[(X-\textbf{E}[X])^2] - 2\textbf{E}[(X-\textbf{E}[X])(Y-\textbf{E}[Y])]+\textbf{E}[(Y-\textbf{E}[Y])^2]$\\
$=\textbf{Var}[X]-\textbf{Cov}[X,Y]+\textbf{Var}[Y]=\textbf{Var}[X]+\textbf{Var}[Y]$ ($X \perp\!\!\!\perp Y$). $\blacksquare$
\section*{3.6}
Let $X_i$ ($1 \leq i \leq k$) be the number of flips after $(i-1)$th head until $i$th head.
Since all flips are independent, the desired variance could be computed as $\sum\limits_{i=1}^k \textbf{Var}[X_i]$.\\
As $X_i\sim Geom(p)$, $\textbf{Var}[X_i]=(1-p)/p^2$ for all $i$. Thus, the desired variance is $k(1-p)/p^2$.
\section*{3.7}
Let $X$ be the number of increases. Then $\Pr(X=k)=\binom{d}{k}p^k(1-p)^{d-k}$.\\
Let the price of the stock after $d$ days be $V$.\\
Then $\textbf{E}[V]=\sum\limits_{k=0}^d qr^k(\frac{1}{r})^{d-k}\binom{d}{k}p^k(1-p)^{d-k}=\sum\limits_{k=0}^dq\binom{d}{k}(pr)^k(\frac{1-p}{r})^{n-k}$.\\
Let $M=pr + (1-p)/r = (1-p+pr^2)/r$. Then\\
$\textbf{E}[V]=M^d\sum\limits_{k=0}^dq\binom{d}{k}(\frac{pr}{M})^k(\frac{1-p}{rM})^{d-k}=M^d\sum\limits_{k=0}^dq\binom{d}{k}(\frac{pr^2}{rM})^k(\frac{1-p}{rM})^{d-k}=M^dq$.\\
Now we compute $\textbf{E}[V^2]=\sum\limits_{k=0}^dq^2 r^{2k}(\frac{1}{r})^{2d-2k}\binom{d}{k}p^k(1-p)^{d-k}$.\\
$\textbf{E}[V^2]=q^2\sum\limits_{k=0}^d\binom{d}{k}(pr^2)^k(\frac{1-p}{r^2})^{d-k}=q^2\left(pr^2+\frac{1-p}{r^2} \right)^d$ (similar to $\textbf{E}[V]$).\\
Thus, $\textbf{Var}[V]=q^2 \left((pr^2 + \frac{1-p}{r^2})^d - (pr + \frac{1-p}{r})^{2d}\right)$.\\
By plugging $q=1$ in, we get the desired result.
\section*{3.8}
Let $X$ be the running time of the given algorithm on input strings of size $n$.
Now, let $M$ be the longest running time of the algorithm among the input strings of size $n$. Then $\Pr(X\geq M) \geq 1/2^n$ by definition.\\
By Markov's inequality, $1/2^n \leq \Pr(X\geq M) \leq \frac{\textbf{E}[X]}{M}$, which leads to $M \leq 2^n \textbf{E}[X]$.
Since $\textbf{E}[X]=O(n^2)$, we get $M=O(n^2 2^n)$.
\section*{3.9}
(a) By linearity of expectations, $\textbf{E}[X^2]=\textbf{E}[\sum\limits_{i=1}^nX_i X]=\sum\limits_{i=1}^n\textbf{E}[X_iX]$.\\
Since $X_i$ are Bernoulli random variables, $\textbf{E}[X_iX]=\Pr(X_i=0)\times 0+\Pr(X_i=1)\times \textbf{E}[X|X_i=1]$. $\blacksquare$\\
(b) Using the equation proven in (a), $\textbf{E}[X^2]=\sum\limits_{i=1}^np\times(1+(n-1)p)=np+n(n-1)p^2$.
Thus, $\textbf{Var}[X]=\textbf{E}[X^2]-(\textbf{E}[X])^2=np+n(n-1)p^2-n^2p^2=np(1-p)$.
\section*{3.10}

\end{document}
