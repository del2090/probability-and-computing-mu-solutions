\documentclass{article}
\usepackage{graphicx} % Required for inserting images
\usepackage{amsmath}
\usepackage{amssymb}

\title{M$\&$U(2nd Edition) Chapter 2 Solutions}
\author{Hahndeul Kim}
\date{January 2025}

\begin{document}

\maketitle
\newpage
\section*{2.1}
$\textbf{E}[X]=\left(\sum\limits_{i=1}^ki\right)/k=(k+1)/2$.
\section*{2.2}
The probability to type "proof" is $1/26^5$. As there are $1,000,000-5+1=999,996$ positions to start the word "proof", the desired probability would be $999996/26^5$ by the linearity of expectations.
\section*{2.3}
Take $f$ as $f(x)=-x^2$ and $X$ as a random variable with $\Pr(X=1)=\Pr(X=2)=1/2$. Then, $-5/2=\textbf{E}[f(X)]<f(\textbf{E}[X])=-9/4$.\\
Take $f$ as $f(x)=x$ and $X$ as above. Then, $\textbf{E}[f(X)]=f(\textbf{E}[X])=3/2$.\\
Take $f$ as $f(x)=x^2$ and $X$ as above. Then, $9/4=f(\textbf{E}[X])<\textbf{E}[f(X)]=5/2$.
\section*{2.4}
Take $f(x)=x^k$, which is convex when $k$ is an positive even integer.
Then by Jensen's inequality, $\textbf{E}[f(X)] \geq f(\textbf{E}[X])$ holds.
\section*{2.5}
Let the event that $X$ is even be $Y$. Then $\Pr(Y)=\sum\limits_{i=0,2,\cdots}\binom{n}{i}({{1}\over{2}})^n$ holds.\\
As is known, $\sum\limits_{i=0,2,\cdots}\binom{n}{i}=2^{n-1}$, so $\Pr(Y)={{1}\over{2}}$ is valid.
\section*{2.6}
\end{document}
