\documentclass{article}
\usepackage{graphicx} % Required for inserting images
\usepackage{amsmath}
\usepackage{amssymb}

\title{M$\&$U(2nd Edition) Chapter 2 Solutions}
\author{Hahndeul Kim}
\date{January 2025}

\begin{document}

\maketitle
\newpage
\section*{2.1}
$\textbf{E}[X]=\left(\sum\limits_{i=1}^ki\right)/k=(k+1)/2$.
\section*{2.2}
The probability to type "proof" is $1/26^5$.
As there are $1,000,000-5+1=999,996$ positions to start the word "proof", the desired probability would be $999996/26^5$ by the linearity of expectations.
\section*{2.3}
Take $f$ as $f(x)=-x^2$ and $X$ as a random variable with $\Pr(X=1)=\Pr(X=2)=1/2$. Then, $-5/2=\textbf{E}[f(X)]<f(\textbf{E}[X])=-9/4$.\\
Take $f$ as $f(x)=x$ and $X$ as above. Then, $\textbf{E}[f(X)]=f(\textbf{E}[X])=3/2$.\\
Take $f$ as $f(x)=x^2$ and $X$ as above. Then, $9/4=f(\textbf{E}[X])<\textbf{E}[f(X)]=5/2$.
\section*{2.4}
Take $f(x)=x^k$, which is convex when $k$ is an positive even integer.
Then by Jensen's inequality, $\textbf{E}[f(X)] \geq f(\textbf{E}[X])$ holds.
\section*{2.5}
Let the event that $X$ is even be $Y$. Then $\Pr(Y)=\sum\limits_{i=0,2,\cdots}\binom{n}{i}(\frac{1}{2})^n$ holds.\
As is known, $\sum\limits_{i=0,2,\cdots}\binom{n}{i}=2^{n-1}$, so $\Pr(Y)={\frac{1}{2}}$ is valid.
\section*{2.6}
(a) $X_1$ can be $2$, $4$ or $6$. Therefore $\textbf{E}[X|X_1$ is even$] = (3+4+\cdots+8)\times \frac{1}{18}+(5+6+\cdots+10)\times \frac{1}{18}+(7+8+\cdots+12)\times \frac{1}{18}=\frac{15}{2}$.\\
(b) $\textbf{E}[X|X_1 = X_2]=(2+4+6+8+10+12)\times\frac{1}{6}=7$.\\
(c) $\textbf{E}[X_1|X=9]=(3+4+5+6)\times\frac{1}{4}=\frac{9}{2}$.\\
(d) $\textbf{E}[X_1 - X_2|X=k]=0$, since $X_1$ and $X_2$ are independent dice rolls.
\section*{2.7}
(a) $\sum\limits_{k=1}^\infty p(1-p)^{k-1}q(1-q)^{k-1}=pq\cdot\frac{1}{1-(1-p)(1-q)}=\frac{pq}{p+q-pq}$.\\
(b) $\textbf{E}[\max (X,Y)]=\sum\limits_{k=1}^\infty \Pr(X\geq k$ or $Y \geq k)=\sum\limits_{k=1}^\infty(1-\Pr(X<k,Y<k))=\sum\limits_{k=1}^\infty(1-(1-(1-p)^{k-1})(1-(1-q)^{k-1}))$\\
$=\sum\limits_{k=1}^\infty((1-p)^{k-1}+(1-q)^{k-1}-(1-p)^{k-1}(1-q)^{k-1})=\frac{1}{p}+\frac{1}{q}-\frac{1}{p+q-pq}$.\\
(c) $\Pr(\min(X,Y)=k)=\Pr(X=k)\Pr(Y\geq k)+\Pr(Y=k)\Pr(X\geq k)-\Pr(X=Y=k)=(1-p)^{k-1}(1-q)^{k-1}(p+q-pq)$.\\
(d) $\textbf{E}[X|X\leq Y]=\textbf{E}[\min(X,Y)]=1/(p+q-pq)$, since $\min(X,Y)\sim Geom(p+q-pq)$ from the previous problem.
\section*{2.8}
(a) Expected number of girls: $\textbf{E}[G]=1\times\sum\limits_{i=1}^k(\frac{1}{2})^i=1-2^{-k}$.\\
Expected number of boys: $\textbf{E}[B]=(\frac{1}{2})^k\times k+\sum\limits_{i=1}^k(\frac{1}{2})^i\times(i-1)=\frac{2^k-1}{2^k}$.\\
(b) The number of total children now follows $Geom(1/2)$. Thus, $\textbf{E}[G+B]=2$ holds.
Since $\textbf{E}[G]=\lim\limits_{k\rightarrow \infty}\frac{2^k-1}{2^k}=1$ holds using the result of the previous problem, $\textbf{E}[B]=1$.
\section*{2.9}
(a) $\textbf{E}[\max(X_1,X_2)]=\sum\limits_{i=1}^k\frac{i^2-(i-1)^2}{k^2}\times i=\frac{4k^2+3k-1}{6k}$.\\
$\textbf{E}[\min(X_1,X_2)]=\sum\limits_{i=1}^k\frac{(k+1-i)^2-(k-i)^2}{k^2}\times i=\frac{2k^2+3k+1}{6k}$.\\
(b) Since two dice are independent, $\textbf{E}[X_1]=\textbf{E}[X_2]=\frac{k+1}{2}$. Therefore, the claim holds.\\
(c) By the linearity of expectations, $\textbf{E}[\max(X_1,X_2)]+\textbf{E}[\min(X_1,X_2)]=\textbf{E}[\max(X_1,X_2)+\min(X_1,X_2)]$ holds.
Since $\{\max(X_1,X_2),\min(X_1,X_2)\}=\{X_1,X_2\}$, $\textbf{E}[\max(X_1,X_2)+\min(X_1,X_2)]=\textbf{E}[X_1+X_2]=\textbf{E}[X_1]+\textbf{E}[X_2]$ holds, again by the linearity of expectations.
Thus, the claim in the previous problem holds.
\end{document}
